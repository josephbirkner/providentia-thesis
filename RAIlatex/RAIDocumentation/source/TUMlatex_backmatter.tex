% !TeX spellcheck = en_US
\section{Using the Auto-Completion Feature of TeXstudio}%
\label{sec:backmatter_autocompletion}%
If you are using TeXstudio\footnote{\url{https://www.texstudio.org/}} for editing LaTeX documents you may use the auto-completion feature together with this package. For details on how to set up this feature be referred to the GitLab project page\footnote{\url{\RAIcoreProjectWebsite}}.\par%
%
%
\section{LaTeX Coding Style}%
Although everybody has a different style of coding, one should consider following style suggestions in order to keep the code clear and readable.\par%
%
\begin{itemize}\itemsep0pt%
    \item Add a \% (percent sign) at the end of each line. This tells the compiler, that the line ends here and can save compilation time. Also place a percent sign in empty lines to avoid irregular spacing in the compiled document.%
    \item Do not name sub-folders ``aux'', since Windows seems to have a problem with that name (especially if you share your document via git).%
    \item Use comments to structure your code.%
    \item Replace tabs by whitespaces in order to guarantee a uniform display on every system.%
\end{itemize}%
%
\textbf{Important:} Note that some ``bad'' code fragments (like empty lines) not only influence the readability of the code but also may change the output and can lead to irregular spacing.\par%
%
%

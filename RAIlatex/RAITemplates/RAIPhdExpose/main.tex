% !TeX spellcheck = en_US
\documentclass[RAIphdexpose%          style
              ,optCharter%            font
              ,optBiber%              bibliography tool
              ,optBibstyleAlphabetic% bibliography style
              ,optEnglish%            language
              %,optTikzExternalize%   compiles faster for large tikz images
              %,optExzellenz%
              ]{RAIlatex}%
%
% Set paths
\graphicspath{{figures/}}%
\addbibresource{source/literature.bib}%
%
\begin{document}%
% Titlepage
% ---------
\RAIphdexposeTitlePage%
{Your Name}%                                                 your name
{your.email@tum.de}%                                         your e-mail address
{your-phone-number}%                                         your phone number (office)
{Provisional Title}%                                         provisional title of your doctoral project
{\RAInamesProfKnoll}%                                        your supervisor
{\RAIlangGerEng{noch nicht festgelegt}{not specified yet}}%  your mentor
{\RAIutilsDate{1}{1}{2020}}%                                 date of this document
%
% Contents
% --------
% HINTS FROM TUM GRADUATE SCHOOL
% A concept paper is a document in which you define the scholarly framework for your dissertation at an early stage
% and develop a realistic working plan. Since it is very hard to predict the framework conditions as well as your
% own research in the context of scientific work, the concept paper will need to be modified on a regular basis. Ideally,
% the concept paper provides the basis for the dissertation. Depending on the topic, style and preliminary work, a concept
% paper can be approximately 6 to 10 pages long after six months of work.
% If you are completing your doctorate as part of a third-party-funded project, many of the building blocks for your concept
% paper will already be included in the project application / project contract. Especially when it comes to the state of
% science and technology, however, you should always make sure that you have a clear understanding of the overall situation.
%
% HINWEISE DER TUM GRADUATE SCHOOL
% Ein Exposé („Konzeptpapier“) ist ein Dokument, in dem Sie zu einem frühen Zeitpunkt den wissenschaftlichen
% Rahmen Ihrer Dissertation eingrenzen und einen realistischen Arbeitsplan entwickeln. Da im wissenschaftlichen
% Betrieb sowohl die Rahmenbedingungen als auch die eigene Forschung nur sehr begrenzt planbar sind, wird das
% Exposé im weiteren Verlauf regelmäßig modifiziert werden müssen. Im Idealfall liefert das Exposé bereits den
% Grundstock für die Dissertation. Je nach Thema, Stil und Vorarbeiten kann ein Exposé nach einem halben Jahr
% etwa 6 bis 10 Seiten umfassen.
% Wenn Sie im Rahmen eines Drittmittelprojekts promovieren werden Sie viele Bausteine für Ihr Exposé bereits
% im Projektantrag / -vertrag finden. Insbesondere beim Stand der Wissenschaft und Technik sollten Sie sich aber
% auf jeden Fall selbst die Sicherheit erarbeiten, dass der Überblick vollständig und Ihnen bekannt ist.
%
%
\section{\RAIlangGerEng{Thema}{Topic}}%
% HINTS FROM TUM GRADUATE SCHOOL
% Concisely describe (no more than half a page) the issue that you want to work on. The relevance of the topic should be apparent.
%
% HINWEISE DER TUM GRADUATE SCHOOL
% Beschreiben Sie prägnant (nicht mehr als eine halbe Seite) die Fragestellung, die Sie bearbeiten möchten. Erkennbar sollte die Relevanz des Themas sein.
%
\blindtext%
%
%
\section{\RAIlangGerEng{Stand der Wissenschaft und Technik}{State of Science and Technology}}%
% HINTS FROM TUM GRADUATE SCHOOL
% Try to gain as comprehensive an overview of the state of science and technology for your topic as possible,
% and provide sources as evidence. After six months, you should be familiar with the preliminary work at your
% own institute, the work of competing scientific groups and the published state of technology. Make sure you will
% be able to follow the developments in these areas as your work progresses.
% For the literature research, use your colleagues’ citation lists (and read the texts!) as well as performing
% your own free database searches. Correctly and fully understanding how your own work is positioned within the
% international state of science and technology as well as the institute’s research traditions is an important
% element of high-quality academic work. Scholarship that is not innovative is not scholarship..
%
% HINWEISE DER TUM GRADUATE SCHOOL
% Versuchen Sie, ein möglichst umfassendes Bild über den Stand der Wissenschaft und Technik zu Ihrem Thema zu
% erhalten und belegen Sie dies mit Quellen. Sie sollten nach einem halben Jahr sowohl die Vorarbeiten am eigenen
% Institut, die der konkurrierenden wissenschaftlichen Gruppen als auch den publizierten Stand der Technik kennen.
% Sorgen Sie dafür, dass Sie im weiteren Verlauf Ihrer Arbeit die Entwicklung in diesen Bereichen laufend verfolgen
% können.
% Nutzen Sie für die Literaturrecherche sowohl die Zitierlisten Ihrer Kollegen (und lesen Sie diese Arbeiten!) als
% auch eine eigene freie Suche in Datenbanken. Die korrekte und vollständige Einordnung der eigenen Arbeit in den
% internationalen Stand der Wissenschaft und Technik und die Forschungstraditionen des Lehrstuhls ist ein entscheidendes
% Qualitätsmerkmal für wissenschaftliche Arbeiten. Wissenschaft ohne Neuigkeitscharakter ist keine.
%
Citation: \cite{choset2005principles, thrun2005probabilistic}\par%
Reference: \cref{fig:MyImage}\par%
Figures:\par%
\begin{figure}[htb]%
    \centering%
    %
    % Including .png
    \includegraphics[width=40mm]{figures/ImagePNG.png}%
    %
    \hspace*{5mm}%
    %
    % Including .pdf
    \includegraphics[width=40mm]{figures/ImagePDF.pdf}\par%
    %
    % Including .tikz
    \begingroup%
        %\RAItikzExternalizeSkipNext%
        \resizebox{40mm}{!}{\input{figures/ImageTIKZ.tikz}}%
    \endgroup%
    %
    \hspace*{5mm}%
    %
    % Including .pdf_tex
    \begingroup%
        \def\svgwidth{40mm}%
        \fontsize{25}{25}\selectfont%
        \input{figures/ImagePDFTEX.pdf_tex}%
    \endgroup%
    %
    \caption{\RAIlangGerEng{Beschreibung des Bilds}{Image description}.}%
    \label{fig:MyImage}%
\end{figure}%
%
\Blindtext%
%
%
\section{\RAIlangGerEng{Ziel der eigenen Arbeit}{Objective of your own Project}}%
% HINTS FROM TUM GRADUATE SCHOOL
% Describe the aspects/areas in which your contribution will go beyond the state of technology described above.
%
% HINWEISE DER TUM GRADUATE SCHOOL
% Beschreiben Sie, in welchen Aspekten/Bereichen Ihr Beitrag über den oben beschriebenen Stand der Technik hinausgehen soll.
%
\blindtext%
%
%
\section{\RAIlangGerEng{Arbeitsplan und benötigte Ressourcen}{Work Plan and necessary Resources}}%
% HINTS FROM TUM GRADUATE SCHOOL
% Try to create a timeframe for the various sub-projects in your dissertation now. Think about
% where there are intersections with and dependencies on your colleagues’ work, and what topic
% areas could also be addressed by students in the context of final projects. It can be helpful
% to create a “Gantt chart.”
% Describe which methods you will use to test which statements, and why these methods are especially
% suitable. Describe which resources, especially experimental resources, you will need for your task.
%
% HINWEISE DER TUM GRADUATE SCHOOL
% Versuchen Sie, bereits jetzt einen Zeitplan für die verschiedenen Teilprojekte Ihrer Dissertation zu erstellen.
% Überlegen Sie, wo es Schnittstellen zu und Abhängigkeiten von Kollegen gibt und welche Themenbereiche u.U. auch
% von Studenten im Rahmen von Abschlussarbeiten bearbeitet werden können. Hilfreich kann dabei sein, ein sog.
% „Gantt-Chart“ zu erstellen.
% Beschreiben Sie, mit welchen Methoden Sie welche Aussagen prüfen wollen und warum diese Methoden besonders gut
% geeignet sind. Beschreiben Sie, welche Ressourcen, insbesondere auch experimentelle, Sie für Ihre Aufgabe benötigen.
%
\cref{fig:time_schedule}\par%
%
\blindtext%
%
\begin{figure}[htb]%
    \centering%
    \RAIlangGerEng{\begin{ganttchart}[%
	y unit chart=.7cm,%
	canvas/.append style={fill=none, draw=black!5, line width=.75pt},%
	hgrid style/.style={draw=black!5, line width=.75pt},%
	vgrid={*1{draw=black!5, line width=.75pt}},%
	title/.style={draw=none, fill=none},%
	title label font=\bfseries\footnotesize,%
	%title label node/.append style={below=5pt},%
	include title in canvas=false,%
	bar height=0.3,%
	bar label font=\mdseries\footnotesize\color{black!70},%
	bar label node/.append style={left=0.2cm},%
	bar/.append style={draw=none, fill=TUMBlue4},%
	group height=0.3,%
	group label node/.append style={left=0.2cm},%
	group label font=\bfseries\small\color{black},%
	group/.append style={draw=none, fill=TUMBlue},%
	]{1}{24}%
	\gantttitle[title label node/.append style={below left=7pt and 10pt}]{Monat}{1}%
	\gantttitlelist[title label node/.append style={below left=7pt and 10pt}]{1,...,6}{4} \\%
	\ganttgroup{\textbf{AP1:} Aktivität A}{1}{5} \\%
	\ganttbar{Aktivität A11}{1}{3} \\%
	\ganttbar{Aktivität A12}{2}{5} \\%
	\ganttgroup{\textbf{AP2:} Aktivität B}{4}{8} \\%
	\ganttbar{Aktivität A21}{4}{8} \\%
	\ganttgroup{\textbf{AP3:} Aktivität C}{6}{20} \\%
	\ganttbar{Aktivität A31}{6}{17} \\%
	\ganttbar{Aktivität A32}{6}{12} \\%
	\ganttbar{Aktivität A32}{10}{20} \\%
	\ganttgroup{\textbf{AP4:} Aktivität D}{18}{22} \\%
	\ganttbar{Aktivität A41}{18}{20} \\%
	\ganttbar{Aktivität A42}{18}{22} \\%
	\ganttgroup{\textbf{AP5:} Aktivität E}{16}{24} \\%
	\ganttbar{Aktivität A51}{16}{24} %
\end{ganttchart}%
}{\begin{ganttchart}[%
	y unit chart=.7cm,%
	canvas/.append style={fill=none, draw=black!5, line width=.75pt},%
	hgrid style/.style={draw=black!5, line width=.75pt},%
	vgrid={*1{draw=black!5, line width=.75pt}},%
	title/.style={draw=none, fill=none},%
	title label font=\bfseries\footnotesize,%
	%title label node/.append style={below=5pt},%
	include title in canvas=false,%
	bar height=0.3,%
	bar label font=\mdseries\footnotesize\color{black!70},%
	bar label node/.append style={left=0.2cm},%
	bar/.append style={draw=none, fill=TUMBlue4},%
	group height=0.3,%
	group label node/.append style={left=0.2cm},%
	group label font=\bfseries\small\color{black},%
	group/.append style={draw=none, fill=TUMBlue},%
	]{1}{24}%
	\gantttitle[title label node/.append style={below left=7pt and 10pt}]{Month}{1}%
	\gantttitlelist[title label node/.append style={below left=7pt and 10pt}]{1,...,6}{4} \\%
	\ganttgroup{\textbf{WP1:} Activity A}{1}{5} \\%
	\ganttbar{Activity A11}{1}{3} \\%
	\ganttbar{Activity A12}{2}{5} \\%
	\ganttgroup{\textbf{WP2:} Activity B}{4}{8} \\%
	\ganttbar{Activity A21}{4}{8} \\%
	\ganttgroup{\textbf{WP3:} Activity C}{6}{20} \\%
	\ganttbar{Activity A31}{6}{17} \\%
	\ganttbar{Activity A32}{6}{12} \\%
	\ganttbar{Activity A32}{10}{20} \\%
	\ganttgroup{\textbf{WP4:} Activity D}{18}{22} \\%
	\ganttbar{Activity A41}{18}{20} \\%
	\ganttbar{Activity A42}{18}{22} \\%
	\ganttgroup{\textbf{WP5:} Activity E}{16}{24} \\%
	\ganttbar{Activity A51}{16}{24}%
\end{ganttchart}%
}%
    \caption{\RAIlangGerEng{Zeitplan}{Time schedule}.}%
    \label{fig:time_schedule}%
\end{figure}%
%
%
\section{\RAIlangGerEng{Literatur}{Literature}}%
% HINTS FROM TUM GRADUATE SCHOOL
% List of literature used
%
% HINWEISE DER TUM GRADUATE SCHOOL
% Auflistung der verwendeten Literatur
%
{%
    %\sloppy% "Word"-like typesetting in order to improve breaking lines with long URLs/DOIs
    \printbibliography[heading=none]%
}%
%
\end{document}%
% HINTS FROM TUM GRADUATE SCHOOL
% A formal review will take place with confirmation from the Faculty Graduate Center using DocGS.
%
% HINWEISE DER TUM GRADUATE SCHOOL
% Die formelle Prüfung durch das Fakultäts-Graduiertenzentrum Maschinenwesen erfolgt durch die Freigabe auf DocGS.
%
%

% !TeX spellcheck = en_US
\documentclass[RAIstudentexpose%      style
              ,optBiber%              bibliography tool
              ,optBibstyleAlphabetic% bibliography style
              ,optEnglish%            language
              %,optTikzExternalize%   compiles faster for large tikz images
              %,optExzellenz%
              ]{RAIlatex}%
%
% Set paths
\graphicspath{{figures/}}%
\addbibresource{source/literature.bib}%
%
\begin{document}%
% Titlepage
% ---------

\RAIstudentexposeTitlePageMastersThesis{Monocular 3D Traffic Perception Using HD Maps as an Auxiliary Feature}{Joseph M. Birkner}{03704462}{joseph.birkner@tum.de}{\RAInamesProfKnoll}{\RAIutilsToday}%

\section{\RAIlangGerEng{Aufgabenstellung}{Topic}}

This work is conducted within the scope of the \textbf{Providentia++} project. The project aims to provide cars with Digital Twins in the cloud, facilitating enhanced traffic prediction and micro-routing abilities for safer autonomous driving. Under the Providentia project, Sensor packages, encompassing RGB Cameras, Radar, and Lidar, are installed along major roads and intersections. We are evaluating, which minimal Sensor-suite would be suitable for a wider rollout of the sensing infrastructure under minimum cost. 

The "Digital Twin" data-point of a vehicle consists of multiple components: First and foremost, position $\left(x|y|z\right)$, size $\left(w|h|d\right)$ and orientation ($\theta$) are key variables which define a vehicle's spatial state. Since the RGB camera sensor is the cheapest among the installed package, 3D object detection from 2D video is an important area of research within Providentia.

A working initial approach for monocular 3D detection for Providentia++ was developed by XXXX in the scope of a Bachelors thesis, using 3D projection of the lower edge of 2D vehicle segments. This approach works very well for vehicles which are moving in straight lines, as the orientation can be fixed to a constant value. However, more work needs to be done for reliable monocular detection when observing traffic scenes with heavily varying vehicle orientations, especially scenes such as complex intersections.

The goal of this work is to improve the orientation estimation of turning vehicles by exploiting clues about their heading from HD maps of the observed road scene. Both map-matching and heading estimation will also benefit from considering the prior trajectory of a vehicle. The trajectory can be obtained by chaining observations of an individual vehicle across multiple video frames, thereby introducing a time component to the observation. Once bounding box estimates between frames are related through vehicle identities, it will also become possible to stabilize predictions about their spatial state via Kalmann Filters, Recurrent Neural Networks or other methods.

\section{\RAIlangGerEng{Lösungsansatz}{Approach}}

% REMARKS
% Give an overview on the approach you want to use for addressing your topic. Include references to literature when
% explaining your proposal.
%
% ANMERKUNGEN
% Geben Sie einen Überblick über den angedachten Lösungsweg Ihrer Fragestellung. Erklären und untermauern Sie Ihren
% Vorschlag mit entsprechenden Quellenangaben.
%
Citation: \cite{choset2005principles, thrun2005probabilistic}\par%
Reference: \cref{fig:MyImage}\par%
Figures:\par%
\begin{figure}[htb]%
    \centering%
    %
    % Including .png
    \includegraphics[width=40mm]{figures/ImagePNG.png}%
    %
    \hspace*{5mm}%
    %
    % Including .pdf
    \includegraphics[width=40mm]{figures/ImagePDF.pdf}\par%
    %
    % Including .tikz
    \begingroup%
        %\RAItikzExternalizeSkipNext%
        \resizebox{40mm}{!}{\input{figures/ImageTIKZ.tikz}}%
    \endgroup%
    %
    \hspace*{5mm}%
    %
    % Including .pdf_tex
    \begingroup%
        \def\svgwidth{40mm}%
        \fontsize{25}{25}\selectfont%
        \input{figures/ImagePDFTEX.pdf_tex}%
    \endgroup%
    %
    \caption{\RAIlangGerEng{Beschreibung des Bilds}{Image description}.}%
    \label{fig:MyImage}%
\end{figure}%
\Blindtext%
%
%
\section{\RAIlangGerEng{Arbeitsplan und benötigte Ressourcen}{Work Plan and necessary Resources}}%
% REMARKS
% Demonstrate how you plan to proceed with your scientific project to achieve the goal. Try to create a timeframe for
% your thesis now. Consider by what time frame and conditions you are bound and where you are dependent on your
% supervisor. To demonstrate your working plan you may create a “Gantt chart”. Describe which methods you will use to
% evaluate your results and why these methods are suitable. Describe which resources (e.g. experimental
% resources) you will need for your task.
%
% ANMERKUNGEN
% Versuchen Sie, bereits jetzt einen Zeitplan für Ihre Abschlussarbeit zu erstellen. Überlegen Sie, an welche
% terminlichen Rahmenbedingungen Sie gebunden sind und an welchen Stellen es Abhängigkeiten von Ihrem Betreuer gibt.
% Sie können Ihren Zeitplan in einer ein sog. „Gantt-Chart“ illustrieren. Beschreiben Sie auch, Ergebnisse evaluieren
% wollen und warum diese Methoden geeignet sind. Beschreiben Sie, welche Ressourcen (z.B. experimentelle) Sie für Ihre
% Aufgabe benötigen.
%
%
%\Blindtext%
%
%\begin{figure*}[!htb]%
%    \centering%
%    \RAIlangGerEng{\begin{ganttchart}[%
	y unit chart=.7cm,%
	canvas/.append style={fill=none, draw=black!5, line width=.75pt},%
	hgrid style/.style={draw=black!5, line width=.75pt},%
	vgrid={*1{draw=black!5, line width=.75pt}},%
	title/.style={draw=none, fill=none},%
	title label font=\bfseries\footnotesize,%
	%title label node/.append style={below=5pt},%
	include title in canvas=false,%
	bar height=0.3,%
	bar label font=\mdseries\footnotesize\color{black!70},%
	bar label node/.append style={left=0.2cm},%
	bar/.append style={draw=none, fill=TUMBlue4},%
	group height=0.3,%
	group label node/.append style={left=0.2cm},%
	group label font=\bfseries\small\color{black},%
	group/.append style={draw=none, fill=TUMBlue},%
	]{1}{24}%
	\gantttitle[title label node/.append style={below left=7pt and 10pt}]{Monat}{1}%
	\gantttitlelist[title label node/.append style={below left=7pt and 10pt}]{1,...,6}{4} \\%
	\ganttgroup{\textbf{AP1:} Aktivität A}{1}{5} \\%
	\ganttbar{Aktivität A11}{1}{3} \\%
	\ganttbar{Aktivität A12}{2}{5} \\%
	\ganttgroup{\textbf{AP2:} Aktivität B}{4}{8} \\%
	\ganttbar{Aktivität A21}{4}{8} \\%
	\ganttgroup{\textbf{AP3:} Aktivität C}{6}{20} \\%
	\ganttbar{Aktivität A31}{6}{17} \\%
	\ganttbar{Aktivität A32}{6}{12} \\%
	\ganttbar{Aktivität A32}{10}{20} \\%
	\ganttgroup{\textbf{AP4:} Aktivität D}{18}{22} \\%
	\ganttbar{Aktivität A41}{18}{20} \\%
	\ganttbar{Aktivität A42}{18}{22} \\%
	\ganttgroup{\textbf{AP5:} Aktivität E}{16}{24} \\%
	\ganttbar{Aktivität A51}{16}{24} %
\end{ganttchart}%
}{\begin{ganttchart}[%
	y unit chart=.7cm,%
	canvas/.append style={fill=none, draw=black!5, line width=.75pt},%
	hgrid style/.style={draw=black!5, line width=.75pt},%
	vgrid={*1{draw=black!5, line width=.75pt}},%
	title/.style={draw=none, fill=none},%
	title label font=\bfseries\footnotesize,%
	%title label node/.append style={below=5pt},%
	include title in canvas=false,%
	bar height=0.3,%
	bar label font=\mdseries\footnotesize\color{black!70},%
	bar label node/.append style={left=0.2cm},%
	bar/.append style={draw=none, fill=TUMBlue4},%
	group height=0.3,%
	group label node/.append style={left=0.2cm},%
	group label font=\bfseries\small\color{black},%
	group/.append style={draw=none, fill=TUMBlue},%
	]{1}{24}%
	\gantttitle[title label node/.append style={below left=7pt and 10pt}]{Month}{1}%
	\gantttitlelist[title label node/.append style={below left=7pt and 10pt}]{1,...,6}{4} \\%
	\ganttgroup{\textbf{WP1:} Activity A}{1}{5} \\%
	\ganttbar{Activity A11}{1}{3} \\%
	\ganttbar{Activity A12}{2}{5} \\%
	\ganttgroup{\textbf{WP2:} Activity B}{4}{8} \\%
	\ganttbar{Activity A21}{4}{8} \\%
	\ganttgroup{\textbf{WP3:} Activity C}{6}{20} \\%
	\ganttbar{Activity A31}{6}{17} \\%
	\ganttbar{Activity A32}{6}{12} \\%
	\ganttbar{Activity A32}{10}{20} \\%
	\ganttgroup{\textbf{WP4:} Activity D}{18}{22} \\%
	\ganttbar{Activity A41}{18}{20} \\%
	\ganttbar{Activity A42}{18}{22} \\%
	\ganttgroup{\textbf{WP5:} Activity E}{16}{24} \\%
	\ganttbar{Activity A51}{16}{24}%
\end{ganttchart}%
}%
%    \caption{\RAIlangGerEng{Zeitplan}{Time schedule}.}%
%\end{figure*}%
%
% References
\section{\RAIlangGerEng{Literatur}{Literature}}%
% REMARKS
% List of used literature.
%
% ANMERKUNGEN
% Auflistung der verwendeten Literatur
{%
    %\sloppy% "Word"-like typesetting in order to improve breaking lines with long URLs/DOIs
    \printbibliography[heading=none]%
}%
%
%
%
\end{document}%

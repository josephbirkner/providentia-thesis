% In total max. 1 Page!
\RAIstudentthesisAbstract{%
%
% Abstract English:
Due to their low cost and high output information density, monocular RGB cameras are a popular sensor choice for many perception tasks, including 3D object detection. The low cost of the sensor is especially important for the large-scale deployment of road-side infrastructure, as envisioned by the Providentia project. Prior work within the project determined, that a two-stage detection approach using the bottom contour of $2D$ instance masks is viable in a highway setting. However, the applicability of the $2D \rightarrow 3D$ lifting approach via the mask in urban scenarios was unclear. In this work, we propose an augmented L-Shape fitting algorithm which solves the monocular 3D object detection task. The algorithm is augmented using HD maps to inform likely heading values, and tracking, to further improve the yaw value selection. We evaluate our algorithm on the Providentia A9R1 urban scenario dataset. The augmented algorithm improves over basic L-Shape fitting by \textbf{+XX} in \textit{mAP}, \textbf{+XX} in \textit{IoU}, \textbf{-XX$\pi$}  in rotational RMSE and \textbf{-XXm} in positional RMSE. We conclude, that the approach is useful for real-time application in road-side infrastructure sensing tasks.
}{%
%
% Zusammenfassung Deutsch:
Aufgrund ihrer geringen Kosten und hohen Informationsdichte sind monokulare RGB-Kameras eine beliebte Sensorwahl für viele Wahrnehmungsaufgaben, einschließlich der 3D-Objekt-erkennung. Die niedrigen Kosten des Sensors sind besonders wichtig für den großflächigen Einsatz von straßenseitiger Infrastruktur, wie er im Rahmen des Providentia-Projektes vorgesehen ist. Frühere Arbeiten im Rahmen des Projekts ergaben, dass ein zweistufiger Erkennungsansatz, der die untere Kontur von $2D$-Instanzmasken verwendet, für den Einsatz an Autobahnen geeignet ist. Unklar war jedoch die Anwendbarkeit des $2D \rightarrow 3D$ Lifting-Ansatzes in städtischen Szenarien. In dieser Arbeit schlagen wir einen erweiterten L-Shape-Fitting Algorithmus vor, der die Aufgabe der monokularen 3D-Objekterkennung löst. Der Algorithmus wird durch HD-Karten erweitert, um wahrscheinliche Heading-Werte zu ermitteln, und Tracking, um die Selektion der Heading-Werte weiter zu verbessern. Wir evaluieren unseren Algorithmus anhand des Providentia A9R1-Datensatzes für urbane Szenarien. Der erweiterte Algorithmus verbessert das grundlegende L-Shape-Fitting um \textbf{+XX} in \textit{mAP}, \textbf{+XX} in \textit{IoU}, \textbf{-XX}$\pi$ im Rotations-RMSE und \textbf{-XXm} im Positions-RMSE. Wir kommen zu dem Schluss, dass der Ansatz für die Echtzeitanwendung in der straßenseitigen Infrastrukturerfassung geeignet ist.
}%
%
%

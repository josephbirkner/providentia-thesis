% !TeX spellcheck = en_US

\chapter{Conclusion}
\label{ch:conclusion}

In the previous chapter, we evaluated various aspects of our proposed two-stage monocular infrastructure traffic object detection architecture, \textit{InfraDet3d}.
We analyzed the performance of different models on the A9 Testfield dataset, using both detailed object categories and the vehicle super-category.
Focusing on vehicles, the best model for the \texttt{S110-S1} perspective achieves a score of $55.60\%$ ($65.54\%$ mAP), while for the \texttt{S110-S2} perspective, the best model scores $50.90\%$ ($60.69\%$ mAP).
The best model across all object categories and scenes exhibits a score of $40.29\%$ ($38.94\%$ mAP).

Moreover, we provided a thorough ablation study of the architecture across all possible configuration regimes.
This covered the impact of different L-Shape-Fitting augmentations, bottom-contour/size filters, different instance segmentation models, and various filter types.
We also accounted for the synchronization lag between the camera frames and the LIDAR-sensor-base labels.

Despite the comprehensive evaluation, there are some shortcomings that can be addressed.
Firstly, the early version of the \textit{A9R1} dataset used in the evaluation has some issues, as some LIDAR labels are inaccurate, potentially affecting the model performance assessment.
This might also explain the ceiling of $~40\%$ IoU, which we seemingly cannot overcome.
Secondly, ablation studies for the VRU detection algorithm and the position-height regression algorithm are missing, which could provide more valuable insights into the system's performance and help identify areas for improvement.

In summary, our evaluation offers a thorough understanding of the system's performance, its limitations, and the challenges that arise from different camera perspectives and computational constraints.
It may now be concluded without a doubt, that the HD map as an auxiliary bias for the L-Shape-Fitting algorithm offers a very significant improvement, guiding the proposed architecture into the realm of production-readiness.
Without the HD map, the detector does not reach an acceptable orientation error.
Therefore, the research hypothesis is confirmed insofar as concluding, that tracking alone does not substitute the HD map as a bias for deciding vehicle orientations.
However, additional analysis and improvements can be made by addressing the mentioned shortcomings, to ultimately enhance the system's performance and reliability even further.

\par\vspace*{\fill}
\epigraph{``Begin at the beginning," the King said gravely, ``and go on till you come to the end: then stop."}{--- \textup{Lewis Carroll}, Alice in Wonderland}

% !TeX spellcheck = en_US

\chapter{Future Work}
\label{ch:future}

In this chapter, we outline potential avenues for future work to further enhance the performance, capabilities, and robustness of our monocular 3d object detection system.

\section{Improved Non-Maximum Suppression (NMS)}
\label{sec:nms}

A common issue encountered in our system is the overlapping of bounding boxes, particularly for pedestrians inside bicycles or vans inside trucks.
To address this problem, we can explore more advanced non-maximum suppression (NMS) techniques that better handle these cases and suppress overlapping boxes, thus improving the system's overall accuracy and robustness.

\section{Substituting HD Maps}
\label{sec:nomap}

Instead of relying solely on HD maps for obtaining heading information, we can explore deriving this information directly from the optical flow.
This approach could offer a more adaptive and flexible solution to changing environments and potentially improve the system's performance in situations where the HD maps are unavailable or inaccurate.

\section{Extended use of Kalman Filters}
\label{sec:extkalman}

The current implementation of Kalman filters in our system does not include heading information in the state.
This omission renders the current approach non-viable for accurate tracking and prediction de-noising.
Extending the Kalman filter to include heading information would significantly improve its effectiveness in estimating the state of the tracked objects.

\section{Improving Pedestrian/Cyclist Detection}
\label{sec:improvepedcyclist}

The current solution for pedestrian and cyclist (VRU) detection is highly unoptimized, as the focus of this work was mostly directed towards road vehicle pose estimation.
Future work should focus on optimizing and refining the detection algorithm, increasing its accuracy and robustness, and attempting to predict VRU orientations.

\section{Neural Keypoint Estimation}
\label{sec:neuralkeypoints}

The long tail robustness of our system can be improved by incorporating neural keypoint estimation techniques.
This approach would help better handle obstructions, occlusions, and unusual vehicle orientations, ultimately enhancing the system's performance in complex and dynamic traffic scenarios.
It could also unify the VRU/Vehicle detection pipelines, and further improve the system's runtime performance.
This presents itself as a very promising research direction.

\section{Amodal Instance Segmentation}
\label{sec:amodal}

Amodal instance segmentation is a way if predicting instance masks while also completing masks where the detection is cut off or obstructed.
This may be a viable alternative to key-points for object detection and segmentation, as it would be a less intrusive architectural change.
While it may not be as robust as key-points in some cases, it might still present a valid approach to improve our system's overall performance.
By investigating amodal instance segmentation techniques, we can potentially enhance the system's ability to detect and segment objects in challenging traffic scenarios.

\par\vspace*{\fill}
\epigraph{``Optimism is a strategy for making a better future. Because unless you believe that the future can be better, you are unlikely to step up and take responsibility for making it so."}{--- \textup{Noam Chomsky}}

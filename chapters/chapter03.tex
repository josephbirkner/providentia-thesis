% !TeX spellcheck = en_US
\chapter{Examples}%

This chapter shows a number of examples that might be useful when you have not worked with LaTex or TUMlatex before.

\section{Figures}
This paragraph shows the differences of pixel based images and vector graphics.
As you can see in Figure~\cref{fig:MyImage} the \code{*.png} image becomes quite blurry when scaled to large, while the other, vector based images look flawless at any zoom rate.%
%
\begin{figure}[htb]%
    \centering%
    %
    % Including .png
    \includegraphics[width=40mm]{figures/ImagePNG.png}%
    %
    \hspace*{5mm}%
    %
    % Including .pdf
    \includegraphics[width=40mm]{figures/ImagePDF.pdf}\par%
    %
    % Including .tikz
    \begingroup%
    %\RAItikzExternalizeSkipNext%
    \resizebox{40mm}{!}{\input{figures/ImageTIKZ.tikz}}%
    \endgroup%
    %
    \hspace*{5mm}%
    %
    % Including .pdf_tex
    \begingroup%
    \def\svgwidth{40mm}%
    \fontsize{25}{25}\selectfont%
    \input{figures/ImagePDFTEX.pdf_tex}%
    \endgroup%
    %
    \caption{\RAIlangGerEng{Beschreibung des Bilds}{Image description}.}%
    \label{fig:MyImage}%
\end{figure}%

\section{Tables}

Tables can be referred to in the same way as figures.
The four letters after the \code{\textbackslash begin\{table\}} statement indicate the placement of the table environment as listed in table~\ref{tab:placement-options}, where the first letter is the preferred position, the second one the next best alternative and so on.

\begin{table}[htb]%
    \centering%
    \begin{tabular}{ll}
        \toprule
        Letter & Placement \\
        \midrule
        \code{h} & here \\
        \code{t} & top of the page \\
        \code{b} & bottom of the page \\
        \code{p} & standalone page \\
        \bottomrule
    \end{tabular}
    \caption{Placement options for floating environments.}
    \label{tab:placement-options}%
\end{table}

We recommend to use the \code{\textbackslash toprule}, \code{\textbackslash midrule} and \code{\textbackslash bottomrule} commands from the \code{booktabs} package to generate nice looking tables.

\section{Equations}

Besides of LaTeX general math mode, that allows you to put some formulas like $a^2 + b^2 = c^2$ directly in your paragraph, we can also use an enumerated, standalone environment for more complex statements, such as Equation~\ref{eq:equivalence-of-mass-and-energy}:

\begin{equation}\label{eq:equivalence-of-mass-and-energy}
E = m c^2
\end{equation}

\section{Plots}

Plots might appear to be just another kind of images, however, \textit{tikz} allows you to define them right in your LaTeX code, as done in figures~\ref{fig:PlotTwoDim} and \ref{fig:PlotThreeDim}.

\begin{figure}[htb]%
    \centering%
    \input{figures/plot_two_dim.tikz}%
    \caption{Title of the 2D plot (using pgfplots).}%
    \label{fig:PlotTwoDim}%
\end{figure}%
%
\begin{figure}[htb]%
    \centering%
    \input{figures/plot_three_dim.tikz}%
    \caption{Title of the 3D plot (using pgfplots).}%
    \label{fig:PlotThreeDim}%
\end{figure}%

\section{Citations}

This sentence is referring to a single source \cite{besl1992method}.
Sentence number two has a single source with a page number \cite[page 123f.]{chen1992object}
The statement in this last sentence is even supported by multiple sources \cite{choset2005principles, thrun2005probabilistic}.

\section{Sections}%

Of course it is also possible to define deeper levels of sections. Let's have a look at some \code{subsections} and \code{subsubsections}.
However, when writing you thesis, try to avoid deeply nested structures.
%
\subsection{Subsection 1}%
\blindtext%
%
\subsubsection{Subsubsection 1}%
\blindtext%
%
\subsubsection{Subsubsection 2}%
\blindtext%
%
\subsection{Subsection 2}%
\blindtext%
%
\subsection{Subsection 3}%
\blindtext%
%

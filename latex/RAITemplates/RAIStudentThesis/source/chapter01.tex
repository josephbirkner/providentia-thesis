% !TeX spellcheck = en_US
\chapter{About TUMlatex}%

\section{Introduction}
This TUMlatex project represents a cooperation of several chairs of the Technical University of Munich to provide a compact and easy-to-use LaTeX package for both, employees and students.

If you are interested in using this package, please read Section~\ref{quick-start-guide}.

If you are an advanced LaTeX user and want to contribute to this project, please read Section~\ref{guide-for-developers}.

If you are responsible for the LaTeX templates of your chair and you are interested in joining this collaboration, please contact the current maintainer Michael Kreutz (\href{mailto:m.kreutz@tum.de}{m.kreutz@tum.de}).

\section{Quick-Start-Guide for Beginners}\label{quick-start-guide}
This project aims to provide an easy-to-use solution for unexperienced or even first-time LaTeX users. As a first step, please read the installation instructions of Section~\ref{installation}.

\subsection{Installation}\label{installation}

If you are using TUMlatex for the first time, please pass through the following steps carefully.

\subsubsection{Setup your LaTeX Environment}\label{setup-environment}

First of all you have to install a LaTeX environment on your PC.
You may skip this step, if you are using a PC of the university with a pre-installed LaTeX environment (e.g. TUM-PC).

\begin{itemize}
    \item For \textbf{Windows} users: Please install the newest version of \underline{\href{https://miktex.org/howto/install-miktex}{MiKTeX}} (see below for full installation).
    \item For \textbf{Mac} users: Please install the newest version of \underline{\href{https://www.tug.org/mactex/}{MacTeX}} (see below for full installation).
    \item For \textbf{Linux} (Ubuntu, \underline{\href{https://www.tug.org/texlive/}{TeXLive}}) - Option A (outdated version - not recommended): Install the default packages texlive-full and biber via apt-get shipped with your Ubuntu distribution.
    \item For \textbf{Linux} (Ubuntu, \underline{\href{https://www.tug.org/texlive/}{TeXLive}}) - Option B (\textbf{newest version - recommended}): Download the TUMlatex package of your chair (see Section~\ref{get-tumlatex}), unzip it and run the automatic installer script \code{UbuntuInstallTexliveNewest.sh} as root\\(\code{sudo ./UbuntuInstallTexliveNewest.sh}) located in XXXTools (replace XXX by your chair abbreviation).
\end{itemize}

\textit{Note: Without any warranty you may use a different LaTeX environment.}

\vspace{0.3cm}

\textbf{Important:} If the amount of free memory on your PC permits, be sure to install all packages.
Otherwise it is likely, that you are missing some necessary packages and have to install them on your own.

\begin{itemize}
    \item For \textbf{Windows} users: Install all packages via the \underline{\href{https://docs.miktex.org/2.9/manual/pkgmgt.html}{MiKTeX Package Manager}}.
    \item For \textbf{Mac} users: Install the \underline{\href{http://www.tug.org/mactex/mactex-download.html}{Full MacTeX package}}.
    \item For \textbf{Linux} users: Both options (A and B) mentioned above will install a full installation of TeXLive.
\end{itemize}

\subsubsection{Optional: Update your LaTeX Environment}

In some cases this is necessary in order to avoid compilation errors due to outdated standard packages. Again you may skip this step if this is not your private PC.

\begin{itemize}
    \item For Windows users: Please follow \underline{\href{https://miktex.org/howto/update-miktex}{this guide}}.
    \item For Mac users: You can use the graphical updating tool \underline{\href{http://amaxwell.github.io/tlutility/}{TeX Live Utility}} to update your packages.
    \item For Linux users: If you chose option B (see Section~\ref{setup-environment}) open a terminal and update the TeXLive package manager and all packages with the command \code{sudo tlmgr update --self --all --reinstall-forcibly-removed}
\end{itemize}

\subsubsection{Setup your LaTeX Editor}

You can use any editor you like. However, the open-source and cross-platform LaTeX editor \underline{\href{http://www.texstudio.org/}{TeXstudio}} is recommended. In this case you can use the \textbf{auto-completion feature} by copying the provided syntax file (\code{.cwl-file}, shipped within the \code{.zip} file, see Section~\ref{get-tumlatex}) to the folder

\begin{itemize}
    \item \code{\%APPDATA\%\textbackslash Roaming\textbackslash texstudio\textbackslash completion\textbackslash user} for Windows users or equivalently
    \item \code{$\sim$/.config/texstudio/} for Linux or Mac users.
\end{itemize}

Furthermore you may enable the shell-escape option of pdflatex (optional, but neccessary for tikz-externalize to work):

\begin{enumerate}
    \item Navigate to \textit{Options $\rightarrow$ Configure TeXstudio $\rightarrow$ Commands}
    \item Change \textit{PdfLaTeX} to \code{pdflatex -synctex=1 -shell-escape -interaction=nonstopmode \%.tex}
\end{enumerate}


You also should change the default bibliography tool from \textit{bibtex} to \textbf{\textit{biber}}:

\begin{enumerate}
    \item Navigate to \textit{Options $\rightarrow$ Configure TeXstudio $\rightarrow$ Build}
    \item Change \textit{Default Bibliography Tool} to \code{txs:///biber}
\end{enumerate}


\textit{Note: You can also use bibtex as bibliography backend with the TUMlatex packages. However, you have to pass \code{optBibtex} instead of \code{optBiber} to the package. Furthermore some commands related to the bibliography are different (ask Google for help).}

\subsubsection{Get the latest Release of the TUMlatex Package}\label{get-tumlatex}

If you are reading this document you probably already have downloaded the TUMlatex package.
However, it is recommended to check if you are using the latest version.
In the following table you find the chair-related TUMlatex packages:

\begin{table}[h]
    \centering
    \begin{tabular}{lll}
        \toprule
        Dept. & Chair & TUMlatex Package \\
        \midrule
        MW & Aerodynamics and Fluid Mechanics & \underline{\href{https://gitlab.lrz.de/AM/TUMlatex/-/jobs/artifacts/release/download?job=ReleaseMWAER}{AERlatex.zip}} \\
        MW & Applied Mechanics & \underline{\href{https://gitlab.lrz.de/AM/TUMlatex/-/jobs/artifacts/release/download?job=ReleaseMWAM}{AMlatex.zip}} \\
        MW & Automotive Technology & \underline{\href{https://wiki.tum.de/display/ftm/Latex-Vorlage}{FTMlatex}} \\
        MW & Machine Tools and Industrial Management & \underline{\href{https://gitlab.lrz.de/AM/TUMlatex/-/jobs/artifacts/release/download?job=ReleaseMWIWB}{IWBlatex.zip}} \\
        MW & Central Teaching Unit & \underline{\href{https://gitlab.lrz.de/AM/TUMlatex/-/jobs/artifacts/release/download?job=ReleaseMWZL}{ZLlatex.zip}} \\
        IN & I6 - Robotics, AI \& Embedded Systems & \underline{\href{https://gitlab.lrz.de/AM/TUMlatex/-/jobs/artifacts/release/download?job=ReleaseINRAI}{RAIlatex.zip}} \\
        \bottomrule
    \end{tabular}
\end{table}

The packages include (replace "\code{XXX}" by your chair abbreviation)

\begin{itemize}
    \item the main class file \code{XXXlatex.cls},
    \item the main package file \code{XXXlatex.sty}
    (This is an alternative to using the main class file.
    This is helpful, if you write a paper and have to use the baseclass provided by the publisher),
    \item the syntax file \code{XXXlatex.cwl} for auto-completion in \underline{\href{http://www.texstudio.org/}{TeXstudio}},
    \item a license file \code{XXXlatex\_LICENSE.txt} containing licensing information,
    \item a folder \code{XXXDocumentation} containing the package documentation (therein precompiled as \code{main.pdf}),
    \item a folder \code{XXXTemplates} containing a selection of useful templates which you can copy and directly use.
    \item a folder \code{XXXTools} containing a selection of useful tools (e.g. TeXLive setup script).
\end{itemize}


\subsection{Create your own Document}

The \textbf{recommended way} to create a new document is to
\begin{enumerate}
    \item select an appropriate template in the template folder \code{XXXTemplates},
    \item \textbf{copy the whole template folder} (e.g. \code{XXXDocument}) wherever you want,
    \item open the \code{main.tex} file in the template folder with your favorite LaTeX-editor,
    \item initially compile it (using \textit{pdflatex} and optionally \textit{biber/bibtex}) and check for errors,
    \item make your changes and fill it with content and
    \item compile it (using \textit{pdflatex} and optionally \textit{biber/bibtex}).
\end{enumerate}

\textit{
    Note: As an alternative way it is possible to use the main class (\code{.cls}-file) or the main package (\code{.sty}-file) as a shared resource by copying it to your local \code{texmf} folder.
    However, in this case you may not be able to compile old documents, which have been created with a different version of this package.
    Thus it is \textbf{strongly recommended} to make a copy of the main class file (\code{XXXlatex.cls}) for each of your documents!
}

\subsection{Documentation}

The documentation of the packages is included in the folder \code{XXXDocumentation} (precompiled as file \code{main.pdf}).
Feel free to have a look at the source code of the documentation, in order to learn how to do stuff.
Be aware that most of the source code is automatically generated.

\subsection{Bugs, Errors and Suggestions}\label{bugs-errors-suggestions}

If you recognize any bug/error or you have a suggestion to make this package (even) better, please feel free to open an \underline{\href{https://gitlab.lrz.de/AM/TUMlatex/issues}{Issue}}.

\textbf{Important:} Assign the corresponding maintainer of your chair to the issue (see \underline{\href{https://gitlab.lrz.de/AM/TUMlatex}{LRZ GitLab}}).
Expect no support for unassigned issues - they will be deleted immediately!

Please use the default \textbf{Labels} to categorize your issue. Please also provide additional information like your
\begin{itemize}
    \item operating system (+ version),
    \item LaTeX environment (+ version) and
    \item LaTeX editor and (+ version).
\end{itemize}


\section{Guide for Developers/Contributors}\label{guide-for-developers}

For information on how to contribute to this project please have a look at the \underline{\href{https://gitlab.lrz.de/AM/TUMlatex/blob/develop/CONTRIBUTING.md}{Contribution Guide}}.

\section{Support}

For support, please directly contact the responsible package maintainer of you chair (see "maintainer" in Section~\ref{bugs-errors-suggestions}).

\textbf{Important:} If your are not associated with a chair or your chair is not part of the TUMlatex project, please read following note:
\textit{
    We are happy that you use TUMlatex and that you appreciate our effort in making a good LaTeX template for TUM members.
    However, please understand that this project is maintained by various PhD students, who contribute to the template mainly in their spare time (actually we do not get paid for this - its our personal commitment).
    Thus, we limit our "support" to employees and students of our associated chairs.
    Certainly, we are happy for any bug report, however, we can not assist everybody in using this template neither can we respond to individual feature requests.
    We hope you can understand this.
}

\section {License Notes}

Copyright (c) 2021 Technical University of Munich (\url{https://www.tum.de/})

This work may be distributed and/or modified under the conditions of the LaTeX Project Public License, either version 1.3 of this license or (at your option) any later version.
The latest version of this license is in \url{http://www.latex-project.org/lppl.txt} and version 1.3 or later is part of all distributions of LaTeX version 2005/12/01 or later.

This work has the LPPL maintenance status 'maintained'.
The current maintainer of this work is Michael Kreutz (\href{mailto:m.kreutz@tum.de}{m.kreutz@tum.de}).

This work consists of all files of the GitLab repository \url{https://gitlab.lrz.de/AM/TUMlatex}.
%

% !TeX spellcheck = en_US
\chapter{I6 Submission Guidelines}%

This chapter gives an overview about the expected format of a thesis at the Chair of Robotics, Artificial Intelligence and Real-time systems.
It mostly focuses on the format on your written thesis.
For a general overview of the submission guideline please refer to the  \underline{\href{https://wiki.tum.de/display/I6intern/Thesis+Submission+Guidelines}{I6 Wiki}}.

When writing your thesis please use the \code{RAIStudentThesis} template.
It should be in line with the IN TUM thesis guidelines (last checked: 08/02/2020).
However, please always reassure that this is still the case. In other subjects, please make sure that the format meets your individual requirements.
When handing it in, please include an additional copy for your advisor. Moreover make sure your that thesis fulfills the following requirements.

\section{Citations}

We recommend that you use \href{https://ctan.org/texarchive/bibliography/bibtex/contrib/german/din1505}{\code{alphadin}} citation style when working with bibtex, or \code{alpha} when using biblatex/biber. Your bibliography should look like this:

\begin{table}[htb]
    \begin{tabular}{lp{13.5cm}}
        $[$BM92$]$ & Besl, P. J. and McKay, N. D. “Method for registration of 3-D shapes”. In: Sensor
        fusion IV: control paradigms and data structures. Vol. 1611. International Society
        for Optics and Photonics. 1992, pp. 586–606.\\
        $[$CM92$]$ & Chen, Y. and Medioni, G. “Object modelling by registration of multiple range
        images”. In: Image and vision computing 10.3 (1992), pp. 145–155.\\
    \end{tabular}
\end{table}

When quoting an author word-by-word please include a page number in the reference.

\section{Figures}

Please use vector graphics when including figures or diagrams in your thesis.
If you create them with \href{https://en.wikipedia.org/wiki/PGF/TikZ}{PGF/TIKZ} or \href{https://inkscape.org/}{Inkscape} it also possible to add texts directly from Latex to ensure a consistent typesetting.
There is an unwritten rule, saying figures not referenced in your text are unnecessary.

When citing a figure include the reference in the figure's label (including the page number).

\section{Math Notation}

When using formulas or pseudocode use the following notation style:

\begin{table}[htb]
    \begin{tabular}{p{2.9cm}lp{5.1cm}p{3.8cm}}
        \toprule
        \textbf{Type} & \textbf{Example} & \textbf{Code (Math Mode)} & \textbf{Comment} \\
        \midrule

        \multicolumn{4}{l}{\textbf{General items}} \\
        \midrule
        Scalars	& $a$ & \code{a} & lower case, italic \\
        Functions & $\textrm{sin}(x)$ & \code{\textbackslash textrm{sin}(x)} & regular text \\
        Units & 42\,Hz & \code{42\textbackslash,\textbackslash textrm{Hz}} & half-space between number and unit; unit in regular text \\
        Angles & $\alpha$ & \code{\textbackslash alpha} & Greek letters \\
        Absolute value & $\mid a \mid$ & \code{\textbackslash mid a \textbackslash mid} \\
        Modulo & mod & \code{\textbackslash textrm\{mod\}} \\
        \midrule
        \multicolumn{4}{l}{\textbf{Vectors and matrices}}\\
        \midrule
        Vectors	& $\boldsymbol{v}$ & \code{\textbackslash boldsymbol\{v\}} & lower case, italic, bold \\
        Matrices & $\boldsymbol{M}$ & \code{\textbackslash boldsymbol\{M\}} & upper case, italic, bold \\
        Variable indices & $\boldsymbol{x}_i$ & \code{\textbackslash boldsymbol\{x\}\_i} \\
        Static indices & $\boldsymbol{x}_\textrm{min}$ & \code{\textbackslash boldsymbol\{x\}\_\textbackslash textrm\{min\}} \\
        Unit vectors & $\textbf{e}_x$ & \code{\textbackslash textbf\{e\}\_x} \\
        Identity matrices & $\textbf{Id}_n$ & \code{\textbackslash textbf\{Id\}\_n}	\\
        Transpose & $\boldsymbol{x}^\textrm{T}$ & \code{\textbackslash boldsymbol\{x\}\^{}\textbackslash textrm\{T\} } & T in superscript and regular font \\
        Vector product (cross product) & $\times$ & \code{\textbackslash times} \\
        Dot product	& $\cdot$ & \code{\textbackslash cdot} \\
        \midrule
        \multicolumn{4}{l}{\textbf{Sets and sequences}}\\
        \midrule
        Sets & $A = \{ 1, 2, 3 \}$ & \code{A = \textbackslash \{ 1, 2, 3 \textbackslash \}} & upper case, italic \\
        Sequence & $\mathcal{A} = \langle 1,2,3 \rangle$ & \code{\textbackslash mathcal\{A\} = \textbackslash langle 1,2,3 \textbackslash rangle} \\
        Set without	& $A \backslash \{ e \}$ & \code{A \textbackslash backslash \textbackslash \{ e \textbackslash \}} \\
        Unification	& $A \cup \{ e \}$ & \code{A \textbackslash cup \textbackslash \{ e \textbackslash \}} & \\

        \bottomrule
    \end{tabular}
\end{table}

Always use single letters for variable names.

\section{Length}

Stundents often ask for a required number of pages.
However, there are no fixed page limits or requirements for bachelor's or master's theses.
As a point of reference, most bachelor's have around 40 to 60 pages and while master's thesis usually reach 60 to 80 pages (including abstract, table-of-contents, lists of figures, lists of tables and bibliography).
If your thesis is going to be longer than 100 pages, your advisor and/or supervisor will most likely not read all of it.
Detailed information can be found in your module handbook (see \url{campus.tum.de}).

\section{Cover}

The main directory of the \code{RAIStudentThesis} template contains two \code{*.tex}: \code{main.tex} and \code{cover.tex}.
Start by using \code{main.tex} to write your thesis.
Once you have completed your work and you know how large the spine of the book will be you can use \code{cover.tex} to generate a matching cover for your printed copies.

\section{CD or DVD}

\textbf{Note:} A CD or DVD is not required for the official copy you submit to the faculty or your program coordinator.
If in doubt check your program's website!

\vspace{0.5cm}

For your advisor's/ supervisor's copy of your thesis please add a CD or DVD on the last inner page of your thesis including:

\begin{enumerate}
    \item A digital copy of your thesis (including its Latex sources)
    \item The code you programmed
    \item The dataset(s) you used for your evaluation results
\end{enumerate}

As a rule of thumb, the disk should include everything required to reproduce your results.

% Auto-generated module documentation
\section{RAIcore}%
\label{sec:module_RAIcore}%
\begingroup\sffamily\begin{tabular}{|p{4cm}|p{11cm}|}%
\hline\bfseries\RAIlangGerEng{Abhängigkeiten}{Dependencies} & none\\\hline%
\bfseries\RAIlangGerEng{Optionen}{Options} & none\\\hline%
\bfseries\RAIlangGerEng{Geladene Pakete}{Loaded Packages} & inputenc etoolbox xparse silence \\\hline%
\bfseries\RAIlangGerEng{Quelldatei}{Sourcefile} & source/IN/RAI/Modules/RAIcore.sty\\\hline%
\bfseries\RAIlangGerEng{Quelldatei (Basis)}{Sourcefile (parent)} & source/TUM/Modules/TUMcore.sty\\%
\hline%
\bfseries\RAIlangGerEng{Befehle}{Commands} & \scriptsize \hyperref[sec:command_RAIcoreLogInfo]{\textbackslash RAIcoreLogInfo} \hyperref[sec:command_RAIcoreLogWarning]{\textbackslash RAIcoreLogWarning} \hyperref[sec:command_RAIcoreLogError]{\textbackslash RAIcoreLogError} \hyperref[sec:command_RAIcoreRepetition]{\textbackslash RAIcoreRepetition} \hyperref[sec:command_RAIcoreProjectWebsite]{\textbackslash RAIcoreProjectWebsite} \\\hline%
\end{tabular}\endgroup\par%
%
%
\subsection*{\RAIlangGerEng{Beschreibung}{Description}}%
% >>> CONTENTS OF FILE source/TUM/Modules/TUMcore_doc.tex: <<<<<<<<<<<<<<<<<<<<<<<<<<<<<<<<<<<<<<<<<
% Documentation of file TUMcore.sty%
The \RAIdocumentationModule{RAIcore}-module is the most important module since it plays a major role in the building process of the class. It helps filtering warnings and debugging the whole document. For more information related to the commands of this module look up \cref{sec:commands}. The results of the debugging process are written to the log file (\textit{main.log}).\par%
%
There are also basic control structures implemented. You can use the command \hyperref[sec:command_RAIcoreRepetition]{\textbackslash RAIcoreRepetition} as simple implementation of a for-loop.\par%
%
%
%
% >>> CONTENTS OF FILE source/IN/RAI/Modules/RAIcore_doc.tex: <<<<<<<<<<<<<<<<<<<<<<<<<<<<<<<<<<<<<<
% Documentation of file RAIcore.sty
%
%
%
%
%
\section{RAIfont}%
\label{sec:module_RAIfont}%
\begingroup\sffamily\begin{tabular}{|p{4cm}|p{11cm}|}%
\hline\bfseries\RAIlangGerEng{Abhängigkeiten}{Dependencies} & \hyperref[sec:module_RAIcore]{RAIcore} \\\hline%
\bfseries\RAIlangGerEng{Optionen}{Options} & optHelvetica optArial optCharter optComputerModern optCharterMath optComputerModernMath \\\hline%
\bfseries\RAIlangGerEng{Geladene Pakete}{Loaded Packages} & fix-cm fontenc helvet uarial charter mathdesign anyfontsize \\\hline%
\bfseries\RAIlangGerEng{Quelldatei}{Sourcefile} & source/IN/RAI/Modules/RAIfont.sty\\\hline%
\bfseries\RAIlangGerEng{Quelldatei (Basis)}{Sourcefile (parent)} & source/TUM/Modules/TUMfont.sty\\%
\hline%
\bfseries\RAIlangGerEng{Befehle}{Commands} & \scriptsize \hyperref[sec:command_RAIfontSetDocumentFontSizeVI]{\textbackslash RAIfontSetDocumentFontSizeVI} \hyperref[sec:command_RAIfontSetDocumentFontSizeVII]{\textbackslash RAIfontSetDocumentFontSizeVII} \hyperref[sec:command_RAIfontSetDocumentFontSizeVIII]{\textbackslash RAIfontSetDocumentFontSizeVIII} \hyperref[sec:command_RAIfontSetDocumentFontSizeIX]{\textbackslash RAIfontSetDocumentFontSizeIX} \hyperref[sec:command_RAIfontSetDocumentFontSizeX]{\textbackslash RAIfontSetDocumentFontSizeX} \hyperref[sec:command_RAIfontSetDocumentFontSizeXI]{\textbackslash RAIfontSetDocumentFontSizeXI} \hyperref[sec:command_RAIfontSetDocumentFontSizeXII]{\textbackslash RAIfontSetDocumentFontSizeXII} \hyperref[sec:command_RAIfontSetDocumentFontSizeXIII]{\textbackslash RAIfontSetDocumentFontSizeXIII} \hyperref[sec:command_RAIfontSetDocumentFontSizeXIV]{\textbackslash RAIfontSetDocumentFontSizeXIV} \hyperref[sec:command_RAIfontSetDocumentFontSizeXV]{\textbackslash RAIfontSetDocumentFontSizeXV} \hyperref[sec:command_RAIfontSetDocumentFontSizeXVI]{\textbackslash RAIfontSetDocumentFontSizeXVI} \hyperref[sec:command_RAIfontSetDocumentFontSizeXVIII]{\textbackslash RAIfontSetDocumentFontSizeXVIII} \hyperref[sec:command_RAIfontSetDocumentFontSizeXX]{\textbackslash RAIfontSetDocumentFontSizeXX} \hyperref[sec:command_RAIfontSetDocumentFontSizeXXII]{\textbackslash RAIfontSetDocumentFontSizeXXII} \hyperref[sec:command_RAIfontSetDocumentFontSizeXXIV]{\textbackslash RAIfontSetDocumentFontSizeXXIV} \hyperref[sec:command_RAIfontSetDocumentFontSizeXXVIII]{\textbackslash RAIfontSetDocumentFontSizeXXVIII} \hyperref[sec:command_RAIfontSetDocumentFontSizeXXLII]{\textbackslash RAIfontSetDocumentFontSizeXXLII} \hyperref[sec:command_RAIfontSetDocumentFontSizeXXLVI]{\textbackslash RAIfontSetDocumentFontSizeXXLVI} \hyperref[sec:command_RAIfontSetDocumentFontSizeXL]{\textbackslash RAIfontSetDocumentFontSizeXL} \hyperref[sec:command_RAIfontSetDocumentFontSizeXLIV]{\textbackslash RAIfontSetDocumentFontSizeXLIV} \\\hline%
\end{tabular}\endgroup\par%
%
%
\subsection*{\RAIlangGerEng{Beschreibung}{Description}}%
% >>> CONTENTS OF FILE source/TUM/Modules/TUMfont_doc.tex: <<<<<<<<<<<<<<<<<<<<<<<<<<<<<<<<<<<<<<<<<
% Documentation of file TUMfont.sty%
The module \RAIdocumentationModule{RAIfont} loads all requested fonts and symbols. There are four different ``text'' fonts available. If you do not select one of them \RAIdocumentationOption{optHelvetica}, or a font specified by the chosen style, will be used as default. The same applies to the two math options, which define the ``mathematical'' font. The default ``mathematical'' font is \RAIdocumentationOption{optCharterMath}.\par%
%
\begin{center}%
    \begin{tabular}{|l|ccc|}%
        \hline%
        \textbf{Option} & \textbf{Sans Serif} & \textbf{Roman} & \textbf{Typewriter}\\\hline%
        \RAIdocumentationOption{optHelvetica} & \textbf{Helvetica} (phv) & Charter (bch) & Com. Mod. (cmtt)\\%
        \RAIdocumentationOption{optArial} & \textbf{URW Arial} (ua1) & Charter (bch) & Com. Mod. (cmtt)\\%
        \RAIdocumentationOption{optCharter} & Helvetica (phv) & \textbf{Charter} (bch) & Com. Mod. (cmtt)\\%
        \RAIdocumentationOption{optComputerModern} & Com. Mod. (cmss) & \textbf{Com. Mod.} (cmr) & Com. Mod. (cmtt)\\%
        \hline%
    \end{tabular}%
\end{center}%
%
Note that URW Arial is \textbf{not} included in standard \LaTeX\ distribution like TeXLive, thus it will only be loaded if the option \RAIdocumentationOption{optArial} is given explicitly. You probably have to install this font manually for your system.\par%
%
\textbf{Hint:} You can switch between sans serif, roman and typewriter font easily with the standard LaTeX commands \RAIdocumentationCode{\textbackslash sffamily}, \RAIdocumentationCode{\textbackslash rmfamily} and \RAIdocumentationCode{\textbackslash ttfamily} respectively.\par%
%
%
\subsection*{Font Families}%
The following table shows examples for the different font options.\par%
%
\begin{center}\vspace{-1em}%
    \begin{tabular}[t]{|l|c|}%
    	\hline%
    	Helvetica (sans serif only) & {\fontfamily{phv}\selectfont Example Text 1 2 3 A B C a b c} \\%
    	\hline%
    	Charter (roman only) & {\fontfamily{bch}\selectfont Example Text 1 2 3 A B C a b c} \\%
    	\hline%
    	Computer Modern (sans serif) & {\fontfamily{cmss}\selectfont Example Text 1 2 3 A B C a b c} \\%
    	\hline%
    	Computer Modern (roman) & {\fontfamily{cmr}\selectfont Example Text 1 2 3 A B C a b c} \\%
    	\hline%
    	Computer Modern (typewriter) & {\fontfamily{cmtt}\selectfont Example Text 1 2 3 A B C a b c}\\%
    	\hline%
    \end{tabular}%
\end{center}%
%
%
\subsection*{Fontsizes}%
By using the commands of this module (starting with \RAIdocumentationCode{\textbackslash RAIfontSetDocumentFontSize}) you can change the default font size of the whole document. Note that it is also possible to use these commands to switch font sizes within a document.\par%
%
The package provides following fontsizes%
\begin{center}\vspace{-1em}%
    \begin{tabular}[t]{|c|c|l|}%
    	\hline%
    	\textbf{Command} & \textbf{Font Size} & \textbf{Comment}\\%
    	\hline%
    	\hyperref[sec:command_RAIfontSetDocumentFontSizeVI]{\textbackslash RAIfontSetDocumentFontSizeVI} & 6pt & \textbf{scaled} from standard article-class\\%
    	\hyperref[sec:command_RAIfontSetDocumentFontSizeVII]{\textbackslash RAIfontSetDocumentFontSizeVII} & 7pt & \textbf{scaled} from standard article-class\\%
    	\hyperref[sec:command_RAIfontSetDocumentFontSizeVIII]{\textbackslash RAIfontSetDocumentFontSizeVIII} & 8pt & \textbf{scaled} from standard article-class\\%
    	\hyperref[sec:command_RAIfontSetDocumentFontSizeIX]{\textbackslash RAIfontSetDocumentFontSizeIX} & 9pt & \textbf{scaled} from standard article-class\\%
    	\hyperref[sec:command_RAIfontSetDocumentFontSizeX]{\textbackslash RAIfontSetDocumentFontSizeX} & 10pt & from standard article-class (\textbf{original})\\%
    	\hyperref[sec:command_RAIfontSetDocumentFontSizeXI]{\textbackslash RAIfontSetDocumentFontSizeXI} & 11pt & from standard article-class (\textbf{original})\\%
    	\hyperref[sec:command_RAIfontSetDocumentFontSizeXII]{\textbackslash RAIfontSetDocumentFontSizeXII} & 12pt & from standard article-class (\textbf{original})\\%
    	\hyperref[sec:command_RAIfontSetDocumentFontSizeXIII]{\textbackslash RAIfontSetDocumentFontSizeXIII} & 13pt & \textbf{scaled} from standard article-class\\%
    	\hyperref[sec:command_RAIfontSetDocumentFontSizeXIV]{\textbackslash RAIfontSetDocumentFontSizeXIV} & 14pt & \textbf{scaled} from standard article-class\\%
    	\hyperref[sec:command_RAIfontSetDocumentFontSizeXV]{\textbackslash RAIfontSetDocumentFontSizeXV} & 15pt & \textbf{scaled} from standard article-class\\%
    	\hyperref[sec:command_RAIfontSetDocumentFontSizeXVI]{\textbackslash RAIfontSetDocumentFontSizeXVI} & 16pt & \textbf{scaled} from standard article-class\\%
    	\hyperref[sec:command_RAIfontSetDocumentFontSizeXVIII]{\textbackslash RAIfontSetDocumentFontSizeXVIII} & 18pt & \textbf{scaled} from standard article-class\\%
    	\hyperref[sec:command_RAIfontSetDocumentFontSizeXX]{\textbackslash RAIfontSetDocumentFontSizeXX} & 20pt & \textbf{scaled} from standard article-class\\%
    	\hyperref[sec:command_RAIfontSetDocumentFontSizeXXII]{\textbackslash RAIfontSetDocumentFontSizeXXII} & 22pt & \textbf{scaled} from standard article-class\\%
    	\hyperref[sec:command_RAIfontSetDocumentFontSizeXXIV]{\textbackslash RAIfontSetDocumentFontSizeXXIV} & 24pt & \textbf{scaled} from standard article-class\\%
    	\hyperref[sec:command_RAIfontSetDocumentFontSizeXXVIII]{\textbackslash RAIfontSetDocumentFontSizeXXVIII} & 28pt & \textbf{scaled} from standard article-class\\%
    	\hyperref[sec:command_RAIfontSetDocumentFontSizeXXLII]{\textbackslash RAIfontSetDocumentFontSizeXXLII} & 32pt & \textbf{scaled} from standard article-class\\%
    	\hyperref[sec:command_RAIfontSetDocumentFontSizeXXLVI]{\textbackslash RAIfontSetDocumentFontSizeXXLVI} & 36pt & \textbf{scaled} from standard article-class\\%
    	\hyperref[sec:command_RAIfontSetDocumentFontSizeXL]{\textbackslash RAIfontSetDocumentFontSizeXL} & 40pt & \textbf{scaled} from standard article-class\\%
    	\hyperref[sec:command_RAIfontSetDocumentFontSizeXLIV]{\textbackslash RAIfontSetDocumentFontSizeXLIV} & 44pt & \textbf{scaled} from standard article-class\\%
    	\hline%
    \end{tabular}%
\end{center}%
%
The commands in the following table will be affected by the font size and accordingly adjusted. In this case the commands for the font size \RAIdocumentationCode{11pt} are shown.\par%
%
\begin{center}\vspace{-1em}%
    \begin{tabular}[t]{|c|c|c|c|c|}%
    	\hline%
    	\tiny{A} & \scriptsize{A} & \footnotesize{A} & \small{A} & \normalsize{A} \\%
    	\hline%
    	\verb+\tiny+ & \verb+\scriptsize+ & \verb+\footnotesize+ & \verb+\small+ & \verb+\normalsize+\\%
    	\hline%
    	\large{A} & \Large{A} & \LARGE{A} & \huge{A} & \Huge{A} \\%
    	\hline%
    	\verb+\large+ & \verb+\Large+ & \verb+\LARGE+ & \verb+\huge+ & \verb+\HUGE+ \\%
    	\hline%
    \end{tabular}%
\end{center}%
%
%
%
% >>> CONTENTS OF FILE source/IN/RAI/Modules/RAIfont_doc.tex: <<<<<<<<<<<<<<<<<<<<<<<<<<<<<<<<<<<<<<
% Documentation of file RAIfont.sty
%
%
%
%
%
\section{RAIlang}%
\label{sec:module_RAIlang}%
\begingroup\sffamily\begin{tabular}{|p{4cm}|p{11cm}|}%
\hline\bfseries\RAIlangGerEng{Abhängigkeiten}{Dependencies} & \hyperref[sec:module_RAIcore]{RAIcore} \\\hline%
\bfseries\RAIlangGerEng{Optionen}{Options} & optGerman optEnglish \\\hline%
\bfseries\RAIlangGerEng{Geladene Pakete}{Loaded Packages} & babel csquotes \\\hline%
\bfseries\RAIlangGerEng{Quelldatei}{Sourcefile} & source/IN/RAI/Modules/RAIlang.sty\\\hline%
\bfseries\RAIlangGerEng{Quelldatei (Basis)}{Sourcefile (parent)} & source/TUM/Modules/TUMlang.sty\\%
\hline%
\bfseries\RAIlangGerEng{Befehle}{Commands} & \scriptsize \hyperref[sec:command_RAIlangSwitchToGerman]{\textbackslash RAIlangSwitchToGerman} \hyperref[sec:command_RAIlangSwitchToEnglish]{\textbackslash RAIlangSwitchToEnglish} \hyperref[sec:command_RAIlangGerEng]{\textbackslash RAIlangGerEng} \hyperref[sec:command_RAIlangForceGerman]{\textbackslash RAIlangForceGerman} \hyperref[sec:command_RAIlangForceEnglish]{\textbackslash RAIlangForceEnglish} \hyperref[sec:command_RAIlangTUM]{\textbackslash RAIlangTUM} \hyperref[sec:command_RAIlangDepartmentMW]{\textbackslash RAIlangDepartmentMW} \hyperref[sec:command_RAIlangDepartmentIN]{\textbackslash RAIlangDepartmentIN} \hyperref[sec:command_RAIlangDepartment]{\textbackslash RAIlangDepartment} \hyperref[sec:command_RAIlangChairMWAER]{\textbackslash RAIlangChairMWAER} \hyperref[sec:command_RAIlangChairMWVIB]{\textbackslash RAIlangChairMWVIB} \hyperref[sec:command_RAIlangChairMWAM]{\textbackslash RAIlangChairMWAM} \hyperref[sec:command_RAIlangChairMWAPT]{\textbackslash RAIlangChairMWAPT} \hyperref[sec:command_RAIlangChairMWAIS]{\textbackslash RAIlangChairMWAIS} \hyperref[sec:command_RAIlangChairMWIWB]{\textbackslash RAIlangChairMWIWB} \hyperref[sec:command_RAIlangChairMWIWBLWF]{\textbackslash RAIlangChairMWIWBLWF} \hyperref[sec:command_RAIlangChairMWIWBLBM]{\textbackslash RAIlangChairMWIWBLBM} \hyperref[sec:command_RAIlangChairMWBVT]{\textbackslash RAIlangChairMWBVT} \hyperref[sec:command_RAIlangChairMWLCC]{\textbackslash RAIlangChairMWLCC} \hyperref[sec:command_RAIlangChairMWLES]{\textbackslash RAIlangChairMWLES} \hyperref[sec:command_RAIlangChairMWLFE]{\textbackslash RAIlangChairMWLFE} \hyperref[sec:command_RAIlangChairMWFTM]{\textbackslash RAIlangChairMWFTM} \hyperref[sec:command_RAIlangChairMWFSD]{\textbackslash RAIlangChairMWFSD} \hyperref[sec:command_RAIlangChairMWFML]{\textbackslash RAIlangChairMWFML} \hyperref[sec:command_RAIlangChairMWLHT]{\textbackslash RAIlangChairMWLHT} \hyperref[sec:command_RAIlangChairMWPKM]{\textbackslash RAIlangChairMWPKM} \hyperref[sec:command_RAIlangChairMWLLB]{\textbackslash RAIlangChairMWLLB} \hyperref[sec:command_RAIlangChairMWLLS]{\textbackslash RAIlangChairMWLLS} \hyperref[sec:command_RAIlangChairMWFZG]{\textbackslash RAIlangChairMWFZG} \hyperref[sec:command_RAIlangChairMWMHPC]{\textbackslash RAIlangChairMWMHPC} \hyperref[sec:command_RAIlangChairMWLMT]{\textbackslash RAIlangChairMWLMT} \hyperref[sec:command_RAIlangChairMWLMM]{\textbackslash RAIlangChairMWLMM} \hyperref[sec:command_RAIlangChairMWLNT]{\textbackslash RAIlangChairMWLNT} \hyperref[sec:command_RAIlangChairMWLNM]{\textbackslash RAIlangChairMWLNM} \hyperref[sec:command_RAIlangChairMWPMW]{\textbackslash RAIlangChairMWPMW} \hyperref[sec:command_RAIlangChairMWPE]{\textbackslash RAIlangChairMWPE} \hyperref[sec:command_RAIlangChairMWPTM]{\textbackslash RAIlangChairMWPTM} \hyperref[sec:command_RAIlangChairMWLFA]{\textbackslash RAIlangChairMWLFA} \hyperref[sec:command_RAIlangChairMWLRT]{\textbackslash RAIlangChairMWLRT} \hyperref[sec:command_RAIlangChairMWRT]{\textbackslash RAIlangChairMWRT} \hyperref[sec:command_RAIlangChairMWSES]{\textbackslash RAIlangChairMWSES} \hyperref[sec:command_RAIlangChairMWSPGM]{\textbackslash RAIlangChairMWSPGM} \hyperref[sec:command_RAIlangChairMWTD]{\textbackslash RAIlangChairMWTD} \hyperref[sec:command_RAIlangChairMWTFD]{\textbackslash RAIlangChairMWTFD} \hyperref[sec:command_RAIlangChairMWUTG]{\textbackslash RAIlangChairMWUTG} \hyperref[sec:command_RAIlangChairMWLVK]{\textbackslash RAIlangChairMWLVK} \hyperref[sec:command_RAIlangChairMWWKM]{\textbackslash RAIlangChairMWWKM} \hyperref[sec:command_RAIlangChairMWLWE]{\textbackslash RAIlangChairMWLWE} \hyperref[sec:command_RAIlangChairMWZFP]{\textbackslash RAIlangChairMWZFP} \hyperref[sec:command_RAIlangChairMWZL]{\textbackslash RAIlangChairMWZL} \hyperref[sec:command_RAIlangChairINRAI]{\textbackslash RAIlangChairINRAI} \hyperref[sec:command_RAIlangChairINSSE]{\textbackslash RAIlangChairINSSE} \hyperref[sec:command_RAIlangChair]{\textbackslash RAIlangChair} \hyperref[sec:command_RAIlangresearchassistantMale]{\textbackslash RAIlangresearchassistantMale} \hyperref[sec:command_RAIlangResearchassistantMale]{\textbackslash RAIlangResearchassistantMale} \hyperref[sec:command_RAIlangResearchAssistantMale]{\textbackslash RAIlangResearchAssistantMale} \hyperref[sec:command_RAIlangresearchassistantFemale]{\textbackslash RAIlangresearchassistantFemale} \hyperref[sec:command_RAIlangResearchassistantFemale]{\textbackslash RAIlangResearchassistantFemale} \hyperref[sec:command_RAIlangResearchAssistantFemale]{\textbackslash RAIlangResearchAssistantFemale} \hyperref[sec:command_RAIlangtenuredprofessorMale]{\textbackslash RAIlangtenuredprofessorMale} \hyperref[sec:command_RAIlangTenuredprofessorMale]{\textbackslash RAIlangTenuredprofessorMale} \hyperref[sec:command_RAIlangTenuredProfessorMale]{\textbackslash RAIlangTenuredProfessorMale} \hyperref[sec:command_RAIlangtenuredprofessorFemale]{\textbackslash RAIlangtenuredprofessorFemale} \hyperref[sec:command_RAIlangTenuredprofessorFemale]{\textbackslash RAIlangTenuredprofessorFemale} \hyperref[sec:command_RAIlangTenuredProfessorFemale]{\textbackslash RAIlangTenuredProfessorFemale} \hyperref[sec:command_RAIlangsupervisor]{\textbackslash RAIlangsupervisor} \hyperref[sec:command_RAIlangSupervisor]{\textbackslash RAIlangSupervisor} \hyperref[sec:command_RAIlangsupervision]{\textbackslash RAIlangsupervision} \hyperref[sec:command_RAIlangSupervision]{\textbackslash RAIlangSupervision} \hyperref[sec:command_RAIlangexaminer]{\textbackslash RAIlangexaminer} \hyperref[sec:command_RAIlangExaminer]{\textbackslash RAIlangExaminer} \hyperref[sec:command_RAIlangbachelorsthesis]{\textbackslash RAIlangbachelorsthesis} \hyperref[sec:command_RAIlangBachelorsthesis]{\textbackslash RAIlangBachelorsthesis} \hyperref[sec:command_RAIlangBachelorsThesis]{\textbackslash RAIlangBachelorsThesis} \hyperref[sec:command_RAIlangdiplomathesis]{\textbackslash RAIlangdiplomathesis} \hyperref[sec:command_RAIlangDiplomathesis]{\textbackslash RAIlangDiplomathesis} \hyperref[sec:command_RAIlangDiplomaThesis]{\textbackslash RAIlangDiplomaThesis} \hyperref[sec:command_RAIlangsemesterthesis]{\textbackslash RAIlangsemesterthesis} \hyperref[sec:command_RAIlangSemesterthesis]{\textbackslash RAIlangSemesterthesis} \hyperref[sec:command_RAIlangSemesterThesis]{\textbackslash RAIlangSemesterThesis} \hyperref[sec:command_RAIlangmastersthesis]{\textbackslash RAIlangmastersthesis} \hyperref[sec:command_RAIlangMastersthesis]{\textbackslash RAIlangMastersthesis} \hyperref[sec:command_RAIlangMastersThesis]{\textbackslash RAIlangMastersThesis} \hyperref[sec:command_RAIlanginterdisciplinaryproject]{\textbackslash RAIlanginterdisciplinaryproject} \hyperref[sec:command_RAIlangInterdisciplinaryproject]{\textbackslash RAIlangInterdisciplinaryproject} \hyperref[sec:command_RAIlangInterdisciplinaryProject]{\textbackslash RAIlangInterdisciplinaryProject} \hyperref[sec:command_RAIlangScientificWorkForObtainingAcademicDegree]{\textbackslash RAIlangScientificWorkForObtainingAcademicDegree} \hyperref[sec:command_RAIlangFieldOfStudyinformatics]{\textbackslash RAIlangFieldOfStudyinformatics} \hyperref[sec:command_RAIlangFieldOfStudyInformatics]{\textbackslash RAIlangFieldOfStudyInformatics} \hyperref[sec:command_RAIlangFieldOfStudygamesengineering]{\textbackslash RAIlangFieldOfStudygamesengineering} \hyperref[sec:command_RAIlangFieldOfStudyGamesengineering]{\textbackslash RAIlangFieldOfStudyGamesengineering} \hyperref[sec:command_RAIlangFieldOfStudyGamesEngineering]{\textbackslash RAIlangFieldOfStudyGamesEngineering} \hyperref[sec:command_RAIlangFieldOfStudyinformationsystems]{\textbackslash RAIlangFieldOfStudyinformationsystems} \hyperref[sec:command_RAIlangFieldOfStudyInformationsystems]{\textbackslash RAIlangFieldOfStudyInformationsystems} \hyperref[sec:command_RAIlangFieldOfStudyInformationSystems]{\textbackslash RAIlangFieldOfStudyInformationSystems} \hyperref[sec:command_RAIlangFieldOfStudybiomedicalcomputing]{\textbackslash RAIlangFieldOfStudybiomedicalcomputing} \hyperref[sec:command_RAIlangFieldOfStudyBiomedicalcomputing]{\textbackslash RAIlangFieldOfStudyBiomedicalcomputing} \hyperref[sec:command_RAIlangFieldOfStudyBiomedicalComputing]{\textbackslash RAIlangFieldOfStudyBiomedicalComputing} \hyperref[sec:command_RAIlangFieldOfStudyautomotivesoftwareengineering]{\textbackslash RAIlangFieldOfStudyautomotivesoftwareengineering} \hyperref[sec:command_RAIlangFieldOfStudyAutomotivesoftwareengineering]{\textbackslash RAIlangFieldOfStudyAutomotivesoftwareengineering} \hyperref[sec:command_RAIlangFieldOfStudyAutomotiveSoftwareEngineering]{\textbackslash RAIlangFieldOfStudyAutomotiveSoftwareEngineering} \hyperref[sec:command_RAIlangFieldOfstudyroboticscognitionintelligence]{\textbackslash RAIlangFieldOfstudyroboticscognitionintelligence} \hyperref[sec:command_RAIlangFieldOfStudyRoboticscognitionintelligence]{\textbackslash RAIlangFieldOfStudyRoboticscognitionintelligence} \hyperref[sec:command_RAIlangFieldOfStudyRoboticsCognitionIntelligence]{\textbackslash RAIlangFieldOfStudyRoboticsCognitionIntelligence} \hyperref[sec:command_RAIlangPhdDegreeDrIng]{\textbackslash RAIlangPhdDegreeDrIng} \hyperref[sec:command_RAIlangPhdDegreeDrIngOld]{\textbackslash RAIlangPhdDegreeDrIngOld} \hyperref[sec:command_RAIlangPhdDegreeDrRerNat]{\textbackslash RAIlangPhdDegreeDrRerNat} \hyperref[sec:command_RAIlangDate]{\textbackslash RAIlangDate} \hyperref[sec:command_RAIlangMonth]{\textbackslash RAIlangMonth} \hyperref[sec:command_RAIlangMonthShort]{\textbackslash RAIlangMonthShort} \hyperref[sec:command_RAIlangmonth]{\textbackslash RAIlangmonth} \hyperref[sec:command_RAIlangmonthShort]{\textbackslash RAIlangmonthShort} \\\hline%
\end{tabular}\endgroup\par%
%
%
\subsection*{\RAIlangGerEng{Beschreibung}{Description}}%
% >>> CONTENTS OF FILE source/TUM/Modules/TUMlang_doc.tex: <<<<<<<<<<<<<<<<<<<<<<<<<<<<<<<<<<<<<<<<<
% Documentation of file TUMlang.sty%
The module \RAIdocumentationModule{RAIlang} gives you the ability to change the language of the document. Thereby, you have the choice between German (\RAIdocumentationOption{optGerman}) and English (\RAIdocumentationOption{optEnglish}). This especially effects predefined headers and titlepages. If no option has been passed, the default language will be German (\RAIdocumentationOption{optGerman}). You can also switch the language of the document dynamically (i.\,e. within the text) with the commands \hyperref[sec:command_RAIlangSwitchToGerman]{\textbackslash RAIlangSwitchToGerman} and \hyperref[sec:command_RAIlangSwitchToEnglish]{\textbackslash RAIlangSwitchToEnglish}.\par%
%
Besides it also includes an accumulation of commands, which displays different words or phrases in English or German. These are all listed in \cref{sec:commands}. Furthermore you can easily create placeholders with the command \hyperref[sec:command_RAIlangGerEng]{\textbackslash RAIlangGerEng} where you can specify german and english translations. Depending on the document language the german or english text will be displayed automatically.\par%
%
%
\subsection*{Dates}%
The module also specifies helper commands to typeset dates in the global document language. You can use the commands \hyperref[sec:command_RAIlangMonth]{\textbackslash RAIlangMonth}, \hyperref[sec:command_RAIlangMonthShort]{\textbackslash RAIlangMonthShort} and \hyperref[sec:command_RAIlangmonth]{\textbackslash RAIlangmonth} to output month names.\par%
%
%
%
% >>> CONTENTS OF FILE source/IN/RAI/Modules/RAIlang_doc.tex: <<<<<<<<<<<<<<<<<<<<<<<<<<<<<<<<<<<<<<
% Documentation of file RAIlang.sty
%
%
%
%
%
\section{RAIaddress}%
\label{sec:module_RAIaddress}%
\begingroup\sffamily\begin{tabular}{|p{4cm}|p{11cm}|}%
\hline\bfseries\RAIlangGerEng{Abhängigkeiten}{Dependencies} & none\\\hline%
\bfseries\RAIlangGerEng{Optionen}{Options} & none\\\hline%
\bfseries\RAIlangGerEng{Geladene Pakete}{Loaded Packages} & none\\\hline%
\bfseries\RAIlangGerEng{Quelldatei}{Sourcefile} & source/IN/RAI/Modules/RAIaddress.sty\\\hline%
\bfseries\RAIlangGerEng{Quelldatei (Basis)}{Sourcefile (parent)} & source/TUM/Modules/TUMaddress.sty\\%
\hline%
\bfseries\RAIlangGerEng{Befehle}{Commands} & \scriptsize \hyperref[sec:command_RAIaddressCityMunich]{\textbackslash RAIaddressCityMunich} \hyperref[sec:command_RAIaddressCityGarching]{\textbackslash RAIaddressCityGarching} \hyperref[sec:command_RAIaddressCityMW]{\textbackslash RAIaddressCityMW} \hyperref[sec:command_RAIaddressCityIN]{\textbackslash RAIaddressCityIN} \hyperref[sec:command_RAIaddressCityDepartment]{\textbackslash RAIaddressCityDepartment} \hyperref[sec:command_RAIaddressCityChair]{\textbackslash RAIaddressCityChair} \hyperref[sec:command_RAIaddressPostalCodeGarching]{\textbackslash RAIaddressPostalCodeGarching} \hyperref[sec:command_RAIaddressPostalCodeMW]{\textbackslash RAIaddressPostalCodeMW} \hyperref[sec:command_RAIaddressPostalCodeIN]{\textbackslash RAIaddressPostalCodeIN} \hyperref[sec:command_RAIaddressPostalCodeDepartment]{\textbackslash RAIaddressPostalCodeDepartment} \hyperref[sec:command_RAIaddressPostalCodeChair]{\textbackslash RAIaddressPostalCodeChair} \hyperref[sec:command_RAIaddressStreetMW]{\textbackslash RAIaddressStreetMW} \hyperref[sec:command_RAIaddressStreetIN]{\textbackslash RAIaddressStreetIN} \hyperref[sec:command_RAIaddressStreetDepartment]{\textbackslash RAIaddressStreetDepartment} \hyperref[sec:command_RAIaddressStreetChair]{\textbackslash RAIaddressStreetChair} \hyperref[sec:command_RAIaddressIBANTUM]{\textbackslash RAIaddressIBANTUM} \hyperref[sec:command_RAIaddressIBANMW]{\textbackslash RAIaddressIBANMW} \hyperref[sec:command_RAIaddressIBANIN]{\textbackslash RAIaddressIBANIN} \hyperref[sec:command_RAIaddressIBANDepartment]{\textbackslash RAIaddressIBANDepartment} \hyperref[sec:command_RAIaddressIBANChair]{\textbackslash RAIaddressIBANChair} \hyperref[sec:command_RAIaddressBICTUM]{\textbackslash RAIaddressBICTUM} \hyperref[sec:command_RAIaddressBICMW]{\textbackslash RAIaddressBICMW} \hyperref[sec:command_RAIaddressBICIN]{\textbackslash RAIaddressBICIN} \hyperref[sec:command_RAIaddressBICDepartment]{\textbackslash RAIaddressBICDepartment} \hyperref[sec:command_RAIaddressBICChair]{\textbackslash RAIaddressBICChair} \hyperref[sec:command_RAIaddressCreditInstitutionTUM]{\textbackslash RAIaddressCreditInstitutionTUM} \hyperref[sec:command_RAIaddressCreditInstitutionMW]{\textbackslash RAIaddressCreditInstitutionMW} \hyperref[sec:command_RAIaddressCreditInstitutionIN]{\textbackslash RAIaddressCreditInstitutionIN} \hyperref[sec:command_RAIaddressCreditInstitutionDepartment]{\textbackslash RAIaddressCreditInstitutionDepartment} \hyperref[sec:command_RAIaddressCreditInstitutionChair]{\textbackslash RAIaddressCreditInstitutionChair} \hyperref[sec:command_RAIaddressWebsiteTUM]{\textbackslash RAIaddressWebsiteTUM} \hyperref[sec:command_RAIaddressWebsiteMW]{\textbackslash RAIaddressWebsiteMW} \hyperref[sec:command_RAIaddressWebsiteIN]{\textbackslash RAIaddressWebsiteIN} \hyperref[sec:command_RAIaddressWebsiteDepartment]{\textbackslash RAIaddressWebsiteDepartment} \hyperref[sec:command_RAIaddressWebsiteChair]{\textbackslash RAIaddressWebsiteChair} \hyperref[sec:command_RAIaddressTaxNumberTUM]{\textbackslash RAIaddressTaxNumberTUM} \hyperref[sec:command_RAIaddressTaxNumberMW]{\textbackslash RAIaddressTaxNumberMW} \hyperref[sec:command_RAIaddressTaxNumberIN]{\textbackslash RAIaddressTaxNumberIN} \hyperref[sec:command_RAIaddressTaxNumberDepartment]{\textbackslash RAIaddressTaxNumberDepartment} \hyperref[sec:command_RAIaddressTaxNumberChair]{\textbackslash RAIaddressTaxNumberChair} \hyperref[sec:command_RAIaddressSalesTaxIDTUM]{\textbackslash RAIaddressSalesTaxIDTUM} \hyperref[sec:command_RAIaddressSalesTaxIDMW]{\textbackslash RAIaddressSalesTaxIDMW} \hyperref[sec:command_RAIaddressSalesTaxIDIN]{\textbackslash RAIaddressSalesTaxIDIN} \hyperref[sec:command_RAIaddressSalesTaxIDDepartment]{\textbackslash RAIaddressSalesTaxIDDepartment} \hyperref[sec:command_RAIaddressSalesTaxIDChair]{\textbackslash RAIaddressSalesTaxIDChair} \\\hline%
\end{tabular}\endgroup\par%
%
%
\subsection*{\RAIlangGerEng{Beschreibung}{Description}}%
% >>> CONTENTS OF FILE source/TUM/Modules/TUMaddress_doc.tex: <<<<<<<<<<<<<<<<<<<<<<<<<<<<<<<<<<<<<<
% Documentation of file TUMaddress.sty%
The \RAIdocumentationModule{RAIaddress}-module defines common strings used in contact information like streets, cities, postal codes, IBANs and BICs. Selected commands:%
\begin{center}%
    \begin{tabular}{|l|l|}%
        \hline%
        \textbf{Command} & \textbf{Output}\\%
        \hline%
    	\hyperref[sec:command_RAIaddressCityMunich]{\textbackslash RAIaddressCityMunich} & \RAIaddressCityMunich\\%
    	\hyperref[sec:command_RAIaddressCityGarching]{\textbackslash RAIaddressCityGarching} & \RAIaddressCityGarching\\%
    	\hyperref[sec:command_RAIaddressCityChair]{\textbackslash RAIaddressCityChair} & \RAIaddressCityChair\\%
    	\hyperref[sec:command_RAIaddressPostalCodeChair]{\textbackslash RAIaddressPostalCodeChair} & \RAIaddressPostalCodeChair\\%
    	\hyperref[sec:command_RAIaddressStreetChair]{\textbackslash RAIaddressStreetChair} & \RAIaddressStreetChair\\%
    	\hyperref[sec:command_RAIaddressIBANChair]{\textbackslash RAIaddressIBANChair} & \RAIaddressIBANChair\\%
    	\hyperref[sec:command_RAIaddressBICChair]{\textbackslash RAIaddressBICChair} & \RAIaddressBICChair\\%
    	\hyperref[sec:command_RAIaddressCreditInstitutionChair]{\textbackslash RAIaddressCreditInstitutionChair} & \RAIaddressCreditInstitutionChair\\%
        \hyperref[sec:command_RAIaddressWebsiteTUM]{\textbackslash RAIaddressWebsiteTUM} & \RAIaddressWebsiteTUM\\%
        \hyperref[sec:command_RAIaddressWebsiteDepartment]{\textbackslash RAIaddressWebsiteDepartment} & \RAIaddressWebsiteDepartment\\%
    	\hyperref[sec:command_RAIaddressWebsiteChair]{\textbackslash RAIaddressWebsiteChair} & \RAIaddressWebsiteChair\\%
    	\hyperref[sec:command_RAIaddressTaxNumberChair]{\textbackslash RAIaddressTaxNumberChair} & \RAIaddressTaxNumberChair\\%
    	\hyperref[sec:command_RAIaddressSalesTaxIDChair]{\textbackslash RAIaddressSalesTaxIDChair} & \RAIaddressSalesTaxIDChair\\%
        \hline%
    \end{tabular}%
\end{center}%
%
%
%
% >>> CONTENTS OF FILE source/IN/RAI/Modules/RAIaddress_doc.tex: <<<<<<<<<<<<<<<<<<<<<<<<<<<<<<<<<<<
% Documentation of file RAIaddress.sty%
%
%
%
%
%
\section{RAIcolor}%
\label{sec:module_RAIcolor}%
\begingroup\sffamily\begin{tabular}{|p{4cm}|p{11cm}|}%
\hline\bfseries\RAIlangGerEng{Abhängigkeiten}{Dependencies} & \hyperref[sec:module_RAIcore]{RAIcore} \\\hline%
\bfseries\RAIlangGerEng{Optionen}{Options} & optRGB optCMYK optGray optMonochrome \\\hline%
\bfseries\RAIlangGerEng{Geladene Pakete}{Loaded Packages} & xcolor \\\hline%
\bfseries\RAIlangGerEng{Quelldatei}{Sourcefile} & source/IN/RAI/Modules/RAIcolor.sty\\\hline%
\bfseries\RAIlangGerEng{Quelldatei (Basis)}{Sourcefile (parent)} & source/TUM/Modules/TUMcolor.sty\\%
\hline%
\bfseries\RAIlangGerEng{Befehle}{Commands} & none\\\hline%
\end{tabular}\endgroup\par%
%
%
\subsection*{\RAIlangGerEng{Beschreibung}{Description}}%
% >>> CONTENTS OF FILE source/TUM/Modules/TUMcolor_doc.tex: <<<<<<<<<<<<<<<<<<<<<<<<<<<<<<<<<<<<<<<<
% Documentation of file TUMcolor.sty%
The module \RAIdocumentationModule{RAIcolor} defines the official colors of the TUM according to the coorporate design. The main options of this module are \RAIdocumentationOption{optRGB}, \RAIdocumentationOption{optCMYK}, \RAIdocumentationOption{optGray} and \RAIdocumentationOption{optMonochrome}, which are different color models. Though CMYK is preferable used for print media and RGB for digital documents. Gray and Monochrome are just different variations of black and white. If no option has been passed, the default option \RAIdocumentationOption{optRGB} will be used.\par%
%
The following tables give an overview over the color definitions.\par%
%
\textbf{Default Colors:}%
\begin{center}%
    \begin{tabular}[t]{|l|c|c|c|c|c|c|c|c|c|}%
		\hline%
    	\textbf{Color} & \textbf{R} & \textbf{G} & \textbf{B} & \textbf{C} & \textbf{M} & \textbf{Y} & \textbf{K} & \textbf{Hex} & \textbf{Preview} \\\hline\hline%
        TUMBlue & 0 & 101 & 189 & 100 & 43 & 0 & 0 & \texttt{0x0065BD} & \raisebox{-0.25em}{\tikz \fill [TUMBlue] (0,0) rectangle (1.3cm,1em);} \\\hline%
    	TUMWhite & 255 & 255 & 255 & 0 & 0 & 0 & 0 & \texttt{0xFFFFFF} & \raisebox{-0.25em}{\tikz \fill [TUMWhite] (0,0) rectangle (1.3cm,1em);} \\\hline%
    	TUMBlack & 0 & 0 & 0 & 0 & 0 & 0 & 100 & \texttt{0x000000} & \raisebox{-0.25em}{\tikz \fill [TUMBlack] (0,0) rectangle (1.3cm,1em);} \\\hline%
        %
        TUMBlue1 & 0 & 51 & 89 & 100 & 57 & 12 & 70 & \texttt{0x003359} & \raisebox{-0.25em}{\tikz \fill [TUMBlue1] (0,0) rectangle (1.3cm,1em);} \\\hline%
        TUMBlue2 & 0 & 82 & 147 & 100 & 54 & 4 & 19 & \texttt{0x005293} & \raisebox{-0.25em}{\tikz \fill [TUMBlue2] (0,0) rectangle (1.3cm,1em);} \\\hline%
        TUMGray1 & 51 & 51 & 51 & 0 & 0 & 0 & 80 & \texttt{0x333333} & \raisebox{-0.25em}{\tikz \fill [TUMGray1] (0,0) rectangle (1.3cm,1em);} \\\hline%
        TUMGray2 & 127 & 127 & 127 & 0 & 0 & 0 & 50 & \texttt{0x7F7F7F} & \raisebox{-0.25em}{\tikz \fill [TUMGray2] (0,0) rectangle (1.3cm,1em);} \\\hline%
        TUMGray3 & 204 & 204 & 204 & 0 & 0 & 0 & 20 & \texttt{0xCCCCCC} & \raisebox{-0.25em}{\tikz \fill [TUMGray3] (0,0) rectangle (1.3cm,1em);} \\\hline%
        %
        TUMBlue3 & 100 & 160 & 200 & 65 & 19 & 1 & 4 & \texttt{0x64A0C8} & \raisebox{-0.25em}{\tikz \fill [TUMBlue3] (0,0) rectangle (1.3cm,1em);} \\\hline%
        TUMBlue4 & 152 & 198 & 234 & 42 & 9 & 0 & 0 & \texttt{0x98C6EA} & \raisebox{-0.25em}{\tikz \fill [TUMBlue4] (0,0) rectangle (1.3cm,1em);} \\\hline%
        TUMIvory & 218 & 215 & 203 & 3 & 4 & 14 & 8 & \texttt{0xDAD7CB} & \raisebox{-0.25em}{\tikz \fill [TUMIvory] (0,0) rectangle (1.3cm,1em);} \\\hline%
        TUMOrange & 227 & 114 & 34 & 0 & 65 & 95 & 0 & \texttt{0xE37222} & \raisebox{-0.25em}{\tikz \fill [TUMOrange] (0,0) rectangle (1.3cm,1em);} \\\hline%
        TUMGreen & 162 & 173 & 0 & 35 & 0 & 100 & 20 & \texttt{0xA2AD00} & \raisebox{-0.25em}{\tikz \fill [TUMGreen] (0,0) rectangle (1.3cm,1em);} \\\hline%
    \end{tabular}%
\end{center}%
%
\textbf{Anniversary Colors:} (150 Jahre culture of excellence)\\%
\textit{\footnotesize (Note: These colors may only be used in the anniversary year 2018.)}%
\begin{center}%
    \begin{tabular}[t]{|l|c|c|c|c|c|c|c|c|c|}%
		\hline%
    	\textbf{Color} & \textbf{R} & \textbf{G} & \textbf{B} & \textbf{C} & \textbf{M} & \textbf{Y} & \textbf{K} & \textbf{Hex} & \textbf{Preview} \\\hline\hline%
        TUMAnniMustard & 232 & 200 & 55 & 12 & 18 & 85 & 0 & \texttt{0xE8C837} & \raisebox{-0.25em}{\tikz \fill [TUMAnniMustard] (0,0) rectangle (1.3cm,1em);} \\\hline%
    	TUMAnniMustard2 & 202 & 171 & 41 & 18 & 25 & 90 & 10 & \texttt{0xCAAB29} & \raisebox{-0.25em}{\tikz \fill [TUMAnniMustard2] (0,0) rectangle (1.3cm,1em);} \\\hline%
        %
    	TUMAnniOrange & 247 & 166 & 0 & 0 & 40 & 100 & 0 & \texttt{0xF7A600} & \raisebox{-0.25em}{\tikz \fill [TUMAnniOrange] (0,0) rectangle (1.3cm,1em);} \\\hline%
        TUMAnniOrange2 & 243 & 145 & 0 & 0 & 50 & 100 & 0 & \texttt{0xF39100} & \raisebox{-0.25em}{\tikz \fill [TUMAnniOrange2] (0,0) rectangle (1.3cm,1em);} \\\hline%
        %
        TUMAnniGreen & 188 & 207 & 30 & 35 & 0 & 95 & 0 & \texttt{0xBCCF1E} & \raisebox{-0.25em}{\tikz \fill [TUMAnniGreen] (0,0) rectangle (1.3cm,1em);} \\\hline%
        TUMAnniGreen2 & 162 & 191 & 22 & 45 & 5 & 100 & 0 & \texttt{0xA2BF16} & \raisebox{-0.25em}{\tikz \fill [TUMAnniGreen2] (0,0) rectangle (1.3cm,1em);} \\\hline%
        %
        TUMAnniBlue & 146 & 212 & 241 & 45 & 0 & 3 & 0 & \texttt{0x92D4F1} & \raisebox{-0.25em}{\tikz \fill [TUMAnniBlue] (0,0) rectangle (1.3cm,1em);} \\\hline%
        TUMAnniBlue2 & 91 & 197 & 242 & 60 & 0 & 0 & 0 & \texttt{0x5BC5F2} & \raisebox{-0.25em}{\tikz \fill [TUMAnniBlue2] (0,0) rectangle (1.3cm,1em);} \\\hline%
        %
        TUMAnniPink & 242 & 144 & 149 & 0 & 55 & 30 & 0 & \texttt{0xF29095} & \raisebox{-0.25em}{\tikz \fill [TUMAnniPink] (0,0) rectangle (1.3cm,1em);} \\\hline%
        TUMAnniPink2 & 227 & 130 & 143 & 8 & 60 & 30 & 0 & \texttt{0xE3828F} & \raisebox{-0.25em}{\tikz \fill [TUMAnniPink2] (0,0) rectangle (1.3cm,1em);} \\\hline%
    \end{tabular}%
\end{center}%
%
%
%
% >>> CONTENTS OF FILE source/IN/RAI/Modules/RAIcolor_doc.tex: <<<<<<<<<<<<<<<<<<<<<<<<<<<<<<<<<<<<<
% Documentation of file RAIcolor.sty
%
%
%
%
%
\section{RAItikz}%
\label{sec:module_RAItikz}%
\begingroup\sffamily\begin{tabular}{|p{4cm}|p{11cm}|}%
\hline\bfseries\RAIlangGerEng{Abhängigkeiten}{Dependencies} & \hyperref[sec:module_RAIcore]{RAIcore} \\\hline%
\bfseries\RAIlangGerEng{Optionen}{Options} & optTikzExternalize \\\hline%
\bfseries\RAIlangGerEng{Geladene Pakete}{Loaded Packages} & pdfpages tikz mdframed pgfplots grffile pgfgantt \\\hline%
\bfseries\RAIlangGerEng{Quelldatei}{Sourcefile} & source/IN/RAI/Modules/RAItikz.sty\\\hline%
\bfseries\RAIlangGerEng{Quelldatei (Basis)}{Sourcefile (parent)} & source/TUM/Modules/TUMtikz.sty\\%
\hline%
\bfseries\RAIlangGerEng{Befehle}{Commands} & \scriptsize \hyperref[sec:command_RAItikzExternalizeSkipNext]{\textbackslash RAItikzExternalizeSkipNext} \hyperref[sec:command_RAItikzGrid]{\textbackslash RAItikzGrid} \hyperref[sec:command_RAItikzCircled]{\textbackslash RAItikzCircled} \\\hline%
\end{tabular}\endgroup\par%
%
%
\subsection*{\RAIlangGerEng{Beschreibung}{Description}}%
% >>> CONTENTS OF FILE source/TUM/Modules/TUMtikz_doc.tex: <<<<<<<<<<<<<<<<<<<<<<<<<<<<<<<<<<<<<<<<<
% Documentation of file TUMtikz.sty%
The \RAIdocumentationModule{RAItikz} module implements basic TikZ, pgfplots and pgfgantt funcitonality. The option \RAIdocumentationOption{optTikzExternalize} quickens the building process by storing TikZ-graphics in a separate folder (\textit{figures/tikzoutput}). When rebuilding the document the compiler will take unchanged graphics directly from the external dataset instead of building the graphic again. However, in some cases positioning problems may occur. In this case the module provides a solution by forcing TikZ to skip the externalization of the next graphic by the command \hyperref[sec:command_RAItikzExternalizeSkipNext]{\textbackslash RAItikzExternalizeSkipNext}.\par%
%
Furthermore the module defines the command \hyperref[sec:command_RAItikzGrid]{\textbackslash RAItikzGrid} with which a grid can be drawn:\par%
%
\RAItikzGrid[dotted]{5mm}{5mm}{20}{3}\par%
%
Additionally macros for framing content are defined:%
%
\begin{center}%
    \begin{tabular}{|ll|}%
        \hline%
        \textbf{Command} & \textbf{Example(s)}\\\hline%
        \hyperref[sec:command_RAItikzCircled]{\textbackslash RAItikzCircled} & \RAItikzCircled{A}, \RAItikzCircled{$\vx$}, \RAItikzCircled{1}\\%
        \hline%
    \end{tabular}%
\end{center}%
%
%
%
% >>> CONTENTS OF FILE source/IN/RAI/Modules/RAItikz_doc.tex: <<<<<<<<<<<<<<<<<<<<<<<<<<<<<<<<<<<<<<
% Documentation of file RAItikz.sty
%
%
%
%
%
\section{RAIlogo}%
\label{sec:module_RAIlogo}%
\begingroup\sffamily\begin{tabular}{|p{4cm}|p{11cm}|}%
\hline\bfseries\RAIlangGerEng{Abhängigkeiten}{Dependencies} & \hyperref[sec:module_RAIcore]{RAIcore} \hyperref[sec:module_RAIfont]{RAIfont} \hyperref[sec:module_RAIcolor]{RAIcolor} \hyperref[sec:module_RAIlang]{RAIlang} \hyperref[sec:module_RAItikz]{RAItikz} \\\hline%
\bfseries\RAIlangGerEng{Optionen}{Options} & none\\\hline%
\bfseries\RAIlangGerEng{Geladene Pakete}{Loaded Packages} & none\\\hline%
\bfseries\RAIlangGerEng{Quelldatei}{Sourcefile} & source/IN/RAI/Modules/RAIlogo.sty\\\hline%
\bfseries\RAIlangGerEng{Quelldatei (Basis)}{Sourcefile (parent)} & source/TUM/Modules/TUMlogo.sty\\%
\hline%
\bfseries\RAIlangGerEng{Befehle}{Commands} & \scriptsize \hyperref[sec:command_RAIlogoChair]{\textbackslash RAIlogoChair} \hyperref[sec:command_RAIlogoTercetTUM]{\textbackslash RAIlogoTercetTUM} \hyperref[sec:command_RAIlogoTercetDepartment]{\textbackslash RAIlogoTercetDepartment} \hyperref[sec:command_RAIlogoTercetChair]{\textbackslash RAIlogoTercetChair} \hyperref[sec:command_RAIlogoTUM]{\textbackslash RAIlogoTUM} \hyperref[sec:command_RAIlogoDepartmentMW]{\textbackslash RAIlogoDepartmentMW} \hyperref[sec:command_RAIlogoDepartmentIN]{\textbackslash RAIlogoDepartmentIN} \hyperref[sec:command_RAIlogoDepartment]{\textbackslash RAIlogoDepartment} \hyperref[sec:command_RAIlogoAnniversary]{\textbackslash RAIlogoAnniversary} \\\hline%
\end{tabular}\endgroup\par%
%
%
\subsection*{\RAIlangGerEng{Beschreibung}{Description}}%
% >>> CONTENTS OF FILE source/TUM/Modules/TUMlogo_doc.tex: <<<<<<<<<<<<<<<<<<<<<<<<<<<<<<<<<<<<<<<<<
% Documentation of file TUMlogo.sty%
The \RAIdocumentationModule{RAIlogo} module creates the logos of the university and the department as TikZ pictures. By passing parameters to the commands, the size and the color of the logos is specified.\par%
%
\begin{center}%
    \begin{tabular}[t]{c@{\hspace{1cm}}c@{\hspace{1cm}}c}%
    	\RAIlogoTUM{2cm}{TUMBlue} & \RAIlogoTUM{2cm}{TUMBlack} & \RAIlogoDepartment{2cm}{TUMBlue}\\[1em]%
        \hyperref[sec:command_RAIlogoTUM]{\textbackslash RAIlogoTUM} & \hyperref[sec:command_RAIlogoTUM]{\textbackslash RAIlogoTUM} & \hyperref[sec:command_RAIlogoDepartment]{\textbackslash RAIlogoDepartment}\\%
    \end{tabular}%
\end{center}%
%
Similar to the logos, also the TUM tercets can be printed in a convenient way:%
%
\begin{center}%
    \begin{tabular}[t]{c@{\hspace{1cm}}c@{\hspace{1cm}}c}%
    	\RAIlogoTercetTUM{1cm} & \RAIlogoTercetDepartment{1cm} & \RAIlogoTercetChair{1cm}\\[1em]%
        \hyperref[sec:command_RAIlogoTercetTUM]{\textbackslash RAIlogoTercetTUM} & \hyperref[sec:command_RAIlogoTercetDepartment]{\textbackslash RAIlogoTercetDepartment} & \hyperref[sec:command_RAIlogoTercetChair]{\textbackslash RAIlogoTercetChair}\\%
    \end{tabular}%
\end{center}%
%
In the anniversary year 2018 one may also use the anniversary badge (\hyperref[sec:command_RAIlogoAnniversary]{\textbackslash RAIlogoAnniversary}):%
%
\begin{center}%
    \begin{tabular}[t]{c@{\hspace{1cm}}c@{\hspace{1cm}}c}%
    	\RAIlogoAnniversary[1cm]{TUMAnniMustard2} & \RAIlogoAnniversary[1cm]{TUMAnniOrange2} & \RAIlogoAnniversary[1cm]{TUMAnniGreen2}\\%
        \hyperref[sec:module_RAIcolor]{TUMAnniMustard2} & \hyperref[sec:module_RAIcolor]{TUMAnniOrange2} & \hyperref[sec:module_RAIcolor]{TUMAnniGreen2}\\[1em]%
    	\RAIlogoAnniversary[1cm]{TUMAnniBlue2} & \RAIlogoAnniversary[1cm]{TUMAnniPink2} & \\%
        \hyperref[sec:module_RAIcolor]{TUMAnniBlue2} & \hyperref[sec:module_RAIcolor]{TUMAnniPink2} & \\%
    \end{tabular}%
\end{center}%
%
%
%
% >>> CONTENTS OF FILE source/IN/RAI/Modules/RAIlogo_doc.tex: <<<<<<<<<<<<<<<<<<<<<<<<<<<<<<<<<<<<<<
% Documentation of file RAIlogo.sty
%
%
%
%
%
\section{RAImath}%
\label{sec:module_RAImath}%
\begingroup\sffamily\begin{tabular}{|p{4cm}|p{11cm}|}%
\hline\bfseries\RAIlangGerEng{Abhängigkeiten}{Dependencies} & \hyperref[sec:module_RAIcore]{RAIcore} \\\hline%
\bfseries\RAIlangGerEng{Optionen}{Options} & optLeftEquations optCenterEquations \\\hline%
\bfseries\RAIlangGerEng{Geladene Pakete}{Loaded Packages} & amsmath mathtools siunitx amssymb amsfonts bm \\\hline%
\bfseries\RAIlangGerEng{Quelldatei}{Sourcefile} & source/IN/RAI/Modules/RAImath.sty\\\hline%
\bfseries\RAIlangGerEng{Quelldatei (Basis)}{Sourcefile (parent)} & source/TUM/Modules/TUMmath.sty\\%
\hline%
\bfseries\RAIlangGerEng{Befehle}{Commands} & \scriptsize \hyperref[sec:command_real]{\textbackslash real} \hyperref[sec:command_imag]{\textbackslash imag} \hyperref[sec:command_asin]{\textbackslash asin} \hyperref[sec:command_acos]{\textbackslash acos} \hyperref[sec:command_atan]{\textbackslash atan} \hyperref[sec:command_co]{\textbackslash co} \hyperref[sec:command_atanII]{\textbackslash atanII} \hyperref[sec:command_AtanII]{\textbackslash AtanII} \hyperref[sec:command_sinabr]{\textbackslash sinabr} \hyperref[sec:command_cosabr]{\textbackslash cosabr} \hyperref[sec:command_sign]{\textbackslash sign} \hyperref[sec:command_sgn]{\textbackslash sgn} \hyperref[sec:command_argmin]{\textbackslash argmin} \hyperref[sec:command_argmax]{\textbackslash argmax} \hyperref[sec:command_div]{\textbackslash div} \hyperref[sec:command_grad]{\textbackslash grad} \hyperref[sec:command_curl]{\textbackslash curl} \hyperref[sec:command_rot]{\textbackslash rot} \hyperref[sec:command_dif]{\textbackslash dif} \hyperref[sec:command_Dif]{\textbackslash Dif} \hyperref[sec:command_order]{\textbackslash order} \hyperref[sec:command_Abs]{\textbackslash Abs} \hyperref[sec:command_abs]{\textbackslash abs} \hyperref[sec:command_Norm]{\textbackslash Norm} \hyperref[sec:command_norm]{\textbackslash norm} \hyperref[sec:command_e]{\textbackslash e} \hyperref[sec:command_const]{\textbackslash const} \hyperref[sec:command_konst]{\textbackslash konst} \hyperref[sec:command_laplacian]{\textbackslash laplacian} \hyperref[sec:command_vdiv]{\textbackslash vdiv} \hyperref[sec:command_vgrad]{\textbackslash vgrad} \hyperref[sec:command_vcurl]{\textbackslash vcurl} \hyperref[sec:command_vrot]{\textbackslash vrot} \hyperref[sec:command_hex]{\textbackslash hex} \hyperref[sec:command_bin]{\textbackslash bin} \hyperref[sec:command_dec]{\textbackslash dec} \hyperref[sec:command_bnot]{\textbackslash bnot} \hyperref[sec:command_MR]{\textbackslash MR} \hyperref[sec:command_MN]{\textbackslash MN} \hyperref[sec:command_MZ]{\textbackslash MZ} \hyperref[sec:command_MC]{\textbackslash MC} \hyperref[sec:command_MQ]{\textbackslash MQ} \hyperref[sec:command_Mone]{\textbackslash Mone} \hyperref[sec:command_eqhat]{\textbackslash eqhat} \hyperref[sec:command_eqexcl]{\textbackslash eqexcl} \hyperref[sec:command_eqdef]{\textbackslash eqdef} \hyperref[sec:command_defined]{\textbackslash defined} \hyperref[sec:command_rdefined]{\textbackslash rdefined} \hyperref[sec:command_vec]{\textbackslash vec} \hyperref[sec:command_vnull]{\textbackslash vnull} \hyperref[sec:command_vzero]{\textbackslash vzero} \hyperref[sec:command_vone]{\textbackslash vone} \hyperref[sec:command_vnabla]{\textbackslash vnabla} \hyperref[sec:command_mat]{\textbackslash mat} \hyperref[sec:command_pmat]{\textbackslash pmat} \hyperref[sec:command_bmat]{\textbackslash bmat} \hyperref[sec:command_Bmat]{\textbackslash Bmat} \hyperref[sec:command_vmat]{\textbackslash vmat} \hyperref[sec:command_Vmat]{\textbackslash Vmat} \hyperref[sec:command_dd]{\textbackslash dd} \hyperref[sec:command_vdot]{\textbackslash vdot} \hyperref[sec:command_vddot]{\textbackslash vddot} \hyperref[sec:command_diff]{\textbackslash diff} \hyperref[sec:command_tdiff]{\textbackslash tdiff} \hyperref[sec:command_pdiff]{\textbackslash pdiff} \hyperref[sec:command_tpdiff]{\textbackslash tpdiff} \hyperref[sec:command_Ddiff]{\textbackslash Ddiff} \hyperref[sec:command_tDdiff]{\textbackslash tDdiff} \hyperref[sec:command_rs]{\textbackslash rs} \\\hline%
\end{tabular}\endgroup\par%
%
%
\subsection*{\RAIlangGerEng{Beschreibung}{Description}}%
% >>> CONTENTS OF FILE source/TUM/Modules/TUMmath_doc.tex: <<<<<<<<<<<<<<<<<<<<<<<<<<<<<<<<<<<<<<<<<
% Documentation of file TUMmath.sty%
The \RAIdocumentationModule{RAImath} module defines various math operators, sets of numbers, numeral systems, relations as well as vector and matrix notations. The options \RAIdocumentationOption{optLeftEquations} (default) and \RAIdocumentationOption{optCenterEquations} allows the user to specify the alignment of equations.%
%
%
\subsection*{Operators}%
\begin{tabular}{@{}|ll|ll|ll|ll|}%
	\hline%
	\textbf{Command} & \textbf{Output} & \textbf{Command} & \textbf{Output} & \textbf{Command} & \textbf{Output} & \textbf{Command} & \textbf{Output}\\\hline%
	\hyperref[sec:command_real]{\textbackslash real} & $\real$ & \hyperref[sec:command_sign]{\textbackslash sign} & $\sign$ & \hyperref[sec:command_curl]{\textbackslash curl} & $\curl$ & \hyperref[sec:command_dif]{\textbackslash dif} & $\dif$ \\%
	%
	\hyperref[sec:command_imag]{\textbackslash imag} & $\imag$ & \hyperref[sec:command_sgn]{\textbackslash sgn} & $\sgn$ & \hyperref[sec:command_rot]{\textbackslash rot}  & $\rot$ & \hyperref[sec:command_dif]{\textbackslash Dif} & $\Dif$\\%
	%
	\hyperref[sec:command_asin]{\textbackslash asin} & $\asin$ & \hyperref[sec:command_argmin]{\textbackslash argmin} & $\argmin$ & \hyperref[sec:command_atanII]{\textbackslash atanII} & $\atanII$ & \hyperref[sec:command_order]{\textbackslash order} & $\order$\\%
	%
	\hyperref[sec:command_acos]{\textbackslash acos} & $\acos$ & \hyperref[sec:command_argmax]{\textbackslash argmax} & $\argmax$ & \hyperref[sec:command_AtanII]{\textbackslash AtanII} & $\AtanII$ & & \\%
	%
	\hyperref[sec:command_atan]{\textbackslash atan} & $\atan$ & \hyperref[sec:command_div]{\textbackslash div} & $\div$ & \hyperref[sec:command_sinabr]{\textbackslash sinabr} & $\sinabr$ &  &  \\%
	%
	\hyperref[sec:command_co]{\textbackslash co} & $\co$ & \hyperref[sec:command_grad]{\textbackslash grad} & $\grad$ & \hyperref[sec:command_cosabr]{\textbackslash cosabr} & $\cosabr$ & & \\%
	\hline%
\end{tabular}\par%
%
\subsection*{Miscellaneous}%
\begin{tabular}{@{}|lc|lc|lc|lc|}%
	\hline%
	\textbf{Command} & \textbf{Output} & \textbf{Command} & \textbf{Output} & \textbf{Command} & \textbf{Output} & \textbf{Command} & \textbf{Output}\\\hline%
	\hyperref[sec:command_Abs]{\textbackslash Abs}\{x\} & $\Abs{x}$ & \hyperref[sec:command_Norm]{\textbackslash Norm}{\{x\}} & $\Norm{x}$ & \hyperref[sec:command_konst]{\textbackslash konst} & $\konst$ & \hyperref[sec:command_vgrad]{\textbackslash vgrad} & $\vgrad$ \\%
	%
	\hyperref[sec:command_abs]{\textbackslash abs}\{x\} & $\abs{x}$ & \hyperref[sec:command_norm]{\textbackslash norm}\{x\} & $\norm{x}$ & \hyperref[sec:command_laplacian]{\textbackslash laplacian} & $\laplacian$ & \hyperref[sec:command_vcurl]{\textbackslash vcurl} & $\vcurl$ \\%
	%
	\hyperref[sec:command_e]{\textbackslash e} & $\e$ & \hyperref[sec:command_const]{\textbackslash const} & $\const$ & \hyperref[sec:command_vdiv]{\textbackslash vdiv} & $\vdiv$ & \hyperref[sec:command_vrot]{\textbackslash vrot} & $\vrot$ \\%
	\hline%
\end{tabular}\par%
%
%
\subsection*{Numeral Systems and Sets of Numbers}%
\begin{tabular}{@{}|lc|lc|lc|lc|}%
    \hline%
    \textbf{Command} & \textbf{Output} & \textbf{Command} & \textbf{Output} & \textbf{Command} & \textbf{Output} & \textbf{Command} & \textbf{Output}\\\hline%
    \hyperref[sec:command_hex]{\textbackslash hex}\{123\} & $\hex{123}$ & \hyperref[sec:command_bin]{\textbackslash bin}\{101\} & $\bin{101}$ & \hyperref[sec:command_dec]{\textbackslash dec}\{123\} & $\dec{123}$ & \hyperref[sec:command_bnot]{\textbackslash bnot}\{101\} & $\bnot{101}$\\%
    %
    \hyperref[sec:command_MR]{\textbackslash MR} & $\MR$ & \hyperref[sec:command_MN]{\textbackslash MN} & $\MN$ & \hyperref[sec:command_MZ]{\textbackslash MZ} & $\MZ$ & \hyperref[sec:command_MC]{\textbackslash MC} & $\MC$\\%
    %
    \hyperref[sec:command_MQ]{\textbackslash MQ} & $\MQ$ & \hyperref[sec:command_Mone]{\textbackslash Mone} & $\Mone$ & & & & \\%
    \hline%
\end{tabular}\par%
%
%
\subsection*{Relations}%
\begin{tabular}{@{}|lc|lc|lc|lc|}%
    \hline%
    \textbf{Command} & \textbf{Output} & \textbf{Command} & \textbf{Output} & \textbf{Command} & \textbf{Output} & \textbf{Command} & \textbf{Output}\\\hline%
    \hyperref[sec:command_eqhat]{\textbackslash eqhat} & $\eqhat$ & \hyperref[sec:command_eqexcl]{\textbackslash eqexcl} & $\eqexcl$ & \hyperref[sec:command_eqdef]{\textbackslash eqdef} & $\eqdef$ & \hyperref[sec:command_defined]{\textbackslash defined} & $\defined$\\%
    %
    \hyperref[sec:command_rdefined]{\textbackslash rdefined} & $\rdefined$ & & & & & & \\%
    \hline%
\end{tabular}\par%
%
%
\subsection*{Derivatives}%
\begin{tabular}{@{}|lc|lc|lc|}%
    \hline%
    \textbf{Command} & \textbf{Output} & \textbf{Command} & \textbf{Output} & \textbf{Command} & \textbf{Output} \\\hline%
    \hyperref[sec:command_dd]{\textbackslash dd}\{x\} & $\dd{x}$ & \hyperref[sec:command_vdot]{\textbackslash vdot}\{x\} & $\vdot{x}$ & \hyperref[sec:command_vddot]{\textbackslash vddot}\{x\} & $\vddot{x}$\\%
    %
    \hyperref[sec:command_diff]{\textbackslash diff}{[2]}\{f\}\{x\} & $\diff[2]{f}{x}$ & \hyperref[sec:command_tdiff]{\textbackslash tdiff}{[2]}\{f\}\{x\} & $\tdiff[2]{f}{x}$ & \hyperref[sec:command_pdiff]{\textbackslash pdiff}{[2]}\{f\}\{x\} & $\pdiff[2]{f}{x}$\\%
    %
    \hyperref[sec:command_tpdiff]{\textbackslash tpdiff}{[2]}\{f\}\{x\} & $\tpdiff[2]{f}{x}$ & \hyperref[sec:command_Ddiff]{\textbackslash Ddiff}{[2]}\{f\}\{x\} & $\Ddiff[2]{f}{x}$ & \hyperref[sec:command_tDdiff]{\textbackslash tDdiff}{[2]}\{f\}\{x\} & $\tDdiff[2]{f}{x}$\\%
    \hline%
\end{tabular}\par%
%
%
\subsection*{Vectors and Matrices}%
Vectors and matrices are usually written as \textbf{bold} symbols. You can either use the explicit command \hyperref[sec:command_vec]{\textbackslash vec} or the short and handy form \RAIdocumentationCode{\textbackslash v+<letter>} to typeset vectors. The placeholder \RAIdocumentationCode{<letter>} can be replaced by any lower or upper case latin or greek letter. Example:\par%
%
\vspace{0.5em}%
\begin{tabular}{@{}|lc|lc|}%
    \hline%
    \textbf{Command} & \textbf{Output} & \textbf{Command} & \textbf{Output} \\\hline%
    \RAIdocumentationCode{\textbackslash vec\{a\}} & $\vec{a}$ & \RAIdocumentationCode{\textbackslash va} & $\va$\\%
    \RAIdocumentationCode{\textbackslash vec\{A\}} & $\vec{A}$ & \RAIdocumentationCode{\textbackslash vA} & $\vA$\\%
    %
    \RAIdocumentationCode{\textbackslash vec\{\textbackslash gamma\}} & $\vec{\gamma}$ & \RAIdocumentationCode{\textbackslash vgamma} & $\vgamma$\\%
    \RAIdocumentationCode{\textbackslash vec\{\textbackslash Gamma\}} & $\vec{\Gamma}$ & \RAIdocumentationCode{\textbackslash vGamma} & $\vGamma$\\%
    \hline%
\end{tabular}\par%
%
\vspace{0.5em}%
Moreover there exist abbreviations for the most common matrix notations:\par%
%
\vspace{0.5em}%
\begin{tabular}{@{}|lc|lc|lc|}%
    \hline%
    \textbf{Command} & \textbf{Output} & \textbf{Command} & \textbf{Output} & \textbf{Command} & \textbf{Output}\\\hline%
    \hyperref[sec:command_mat]{\textbackslash mat}\{1\&2\textbackslash\textbackslash 3\&4\} & $\mat{1 & 2\\ 3 & 4}$ & \hyperref[sec:command_pmat]{\textbackslash pmat}\{1\&2\textbackslash\textbackslash 3\&4\} & $\pmat{1 & 2\\ 3 & 4}$ & \hyperref[sec:command_bmat]{\textbackslash bmat}\{1\&2\textbackslash\textbackslash 3\&4\} & $\bmat{1 & 2\\ 3 & 4}$\\[1.5em]%
    %
    \hyperref[sec:command_Bmat]{\textbackslash Bmat}\{1\&2\textbackslash\textbackslash 3\&4\} & $\Bmat{1 & 2\\ 3 & 4}$ & \hyperref[sec:command_vmat]{\textbackslash vmat}\{1\&2\textbackslash\textbackslash 3\&4\} & $\vmat{1 & 2\\ 3 & 4}$ & \hyperref[sec:command_Vmat]{\textbackslash Vmat}\{1\&2\textbackslash\textbackslash 3\&4\} & $\Vmat{1 & 2\\ 3 & 4}$\\%
    \hline%
\end{tabular}\par%
%
%
\subsection*{Indices}%
There exists the auxillary command \hyperref[sec:command_rs]{\textbackslash rs} to typeset leftsided indices in equations. Example: type \RAIdocumentationCode{\textbackslash rs\{\_a\^{}b\}\{\textbackslash vC\}\_c\^{}d} to obtain $\rs{_a^b}{\vX}_c^d$.\par%
%
%
%
% >>> CONTENTS OF FILE source/IN/RAI/Modules/RAImath_doc.tex: <<<<<<<<<<<<<<<<<<<<<<<<<<<<<<<<<<<<<<
% Documentation of file RAImath.sty
%
%
%
%
%
\section{RAIlayout}%
\label{sec:module_RAIlayout}%
\begingroup\sffamily\begin{tabular}{|p{4cm}|p{11cm}|}%
\hline\bfseries\RAIlangGerEng{Abhängigkeiten}{Dependencies} & \hyperref[sec:module_RAIcore]{RAIcore} \hyperref[sec:module_RAIcolor]{RAIcolor} \hyperref[sec:module_RAItikz]{RAItikz} \hyperref[sec:module_RAIlogo]{RAIlogo} \\\hline%
\bfseries\RAIlangGerEng{Optionen}{Options} & optAfive optAfour optAthree optAtwo optAone optAzero optBeamerClassicFormat optBeamerWideFormat optExzellenz \\\hline%
\bfseries\RAIlangGerEng{Geladene Pakete}{Loaded Packages} & geometry \\\hline%
\bfseries\RAIlangGerEng{Quelldatei}{Sourcefile} & source/IN/RAI/Modules/RAIlayout.sty\\\hline%
\bfseries\RAIlangGerEng{Quelldatei (Basis)}{Sourcefile (parent)} & source/TUM/Modules/TUMlayout.sty\\%
\hline%
\bfseries\RAIlangGerEng{Befehle}{Commands} & \scriptsize \hyperref[sec:command_RAIlayoutHeaderCustomChair]{\textbackslash RAIlayoutHeaderCustomChair} \hyperref[sec:command_RAIlayoutTitlePageDefault]{\textbackslash RAIlayoutTitlePageDefault} \hyperref[sec:command_RAIlayoutSetIndent]{\textbackslash RAIlayoutSetIndent} \hyperref[sec:command_RAIlayoutNoIndent]{\textbackslash RAIlayoutNoIndent} \hyperref[sec:command_RAIlayoutSetAfive]{\textbackslash RAIlayoutSetAfive} \hyperref[sec:command_RAIlayoutSetAfour]{\textbackslash RAIlayoutSetAfour} \hyperref[sec:command_RAIlayoutSetAthree]{\textbackslash RAIlayoutSetAthree} \hyperref[sec:command_RAIlayoutSetAtwo]{\textbackslash RAIlayoutSetAtwo} \hyperref[sec:command_RAIlayoutSetAone]{\textbackslash RAIlayoutSetAone} \hyperref[sec:command_RAIlayoutSetAzero]{\textbackslash RAIlayoutSetAzero} \hyperref[sec:command_RAIlayoutSetBeamerClassicFormat]{\textbackslash RAIlayoutSetBeamerClassicFormat} \hyperref[sec:command_RAIlayoutSetBeamerWideFormat]{\textbackslash RAIlayoutSetBeamerWideFormat} \hyperref[sec:command_RAIlayoutPutAtCenter]{\textbackslash RAIlayoutPutAtCenter} \hyperref[sec:command_RAIlayoutPutAtNorthWest]{\textbackslash RAIlayoutPutAtNorthWest} \hyperref[sec:command_RAIlayoutPutAtNorthWestCentered]{\textbackslash RAIlayoutPutAtNorthWestCentered} \hyperref[sec:command_RAIlayoutPutAtNorthEast]{\textbackslash RAIlayoutPutAtNorthEast} \hyperref[sec:command_RAIlayoutPutAtNorthEastCentered]{\textbackslash RAIlayoutPutAtNorthEastCentered} \hyperref[sec:command_RAIlayoutPutAtSouthWest]{\textbackslash RAIlayoutPutAtSouthWest} \hyperref[sec:command_RAIlayoutPutAtSouthWestCentered]{\textbackslash RAIlayoutPutAtSouthWestCentered} \hyperref[sec:command_RAIlayoutPutAtSouthEast]{\textbackslash RAIlayoutPutAtSouthEast} \hyperref[sec:command_RAIlayoutPutAtSouthEastCentered]{\textbackslash RAIlayoutPutAtSouthEastCentered} \hyperref[sec:command_RAIlayoutHeaderCDTUMLogoOnly]{\textbackslash RAIlayoutHeaderCDTUMLogoOnly} \hyperref[sec:command_RAIlayoutHeaderCDTUM]{\textbackslash RAIlayoutHeaderCDTUM} \hyperref[sec:command_RAIlayoutHeaderCDDepartment]{\textbackslash RAIlayoutHeaderCDDepartment} \hyperref[sec:command_RAIlayoutHeaderCDChair]{\textbackslash RAIlayoutHeaderCDChair} \hyperref[sec:command_RAIlayoutCDBorder]{\textbackslash RAIlayoutCDBorder} \\\hline%
\end{tabular}\endgroup\par%
%
%
\subsection*{\RAIlangGerEng{Beschreibung}{Description}}%
% >>> CONTENTS OF FILE source/TUM/Modules/TUMlayout_doc.tex: <<<<<<<<<<<<<<<<<<<<<<<<<<<<<<<<<<<<<<<
% Documentation of file TUMLayout.sty%
The \RAIdocumentationModule{RAIlayout} module defines page size and margins, commands for absolute positioning of content on the page and header definitions. The four options of this module specify the document page format. If no size option is specified the default option \RAIdocumentationOption{optAfour} is used.\par%
%
The following table gives an overview of the page formats.\par%
%
\begin{center}%
    \begin{tabular}{|lll|}%
        \hline%
        \textbf{Option} & \textbf{Size} & \textbf{Margins}\\\hline%
        \RAIdocumentationOption{optAfive} & DIN A5 (148mm $\times$ 210mm) & 1.41cm\\%
        \RAIdocumentationOption{optAfour} & DIN A4 (210mm $\times$ 297mm) & 2cm\\%
        \RAIdocumentationOption{optAthree} & DIN A3 (297mm $\times$ 420mm) & 2.82cm\\%
        \RAIdocumentationOption{optAtwo} & DIN A2 (420mm $\times$ 594mm) & 4cm\\%
        \RAIdocumentationOption{optAone} & DIN A1 (594mm $\times$ 841mm) & 5.66cm\\%
        \RAIdocumentationOption{optAzero} & DIN A0 (841mm $\times$ 1189mm) & 8cm\\%
        \RAIdocumentationOption{optBeamerClassicFormat} & 4:3 (254mm $\times$ 190.5mm) & 1cm (left and right)\\%
        \RAIdocumentationOption{optBeamerWideFormat} & 16:9 (254mm $\times$ 142.9mm) & 1cm (left and right)\\%
        \hline%
    \end{tabular}%
\end{center}%
%
%
\subsection*{Absolute Positioning}%
For absolute positioning on the page several convenience commands exits. They all start with \RAIdocumentationCode{\textbackslash RAIlayoutPutAt} and allow positioning relative to global page markers, i.\,e. the corners or the page center. The content is integrated into a TikZ-picture whose anchor is in the same corner as the specified page corner. Moreover the content may be rotated by any angle of rotation.\par%
%
\textbf{Important:} Note that due to positioning issues TikZ-pictures cannot be externalized in combination with this commands.\par%
%
\subsection*{Headers}%
The module defines several headers which differ in the amount of information displayed. Headers including the letters ``CD'' correspond to the corporate design of the TUM. By passing the option \RAIdocumentationOption{optExzellenz} you can enable the display of the anniversary badge (use this only in the anniversary year 2018).\par%
%
%
%
% >>> CONTENTS OF FILE source/IN/RAI/Modules/RAIlayout_doc.tex: <<<<<<<<<<<<<<<<<<<<<<<<<<<<<<<<<<<<
% Documentation of file RAIlayout.sty
%
%
%
%
%
\section{RAIbiblio}%
\label{sec:module_RAIbiblio}%
\begingroup\sffamily\begin{tabular}{|p{4cm}|p{11cm}|}%
\hline\bfseries\RAIlangGerEng{Abhängigkeiten}{Dependencies} & \hyperref[sec:module_RAIcore]{RAIcore} \hyperref[sec:module_RAIlang]{RAIlang} \\\hline%
\bfseries\RAIlangGerEng{Optionen}{Options} & optBiber optBibtex optBibstyleNumeric optBibstyleAlphabetic optBibstyleAuthorYear \\\hline%
\bfseries\RAIlangGerEng{Geladene Pakete}{Loaded Packages} & biblatex \\\hline%
\bfseries\RAIlangGerEng{Quelldatei}{Sourcefile} & source/IN/RAI/Modules/RAIbiblio.sty\\\hline%
\bfseries\RAIlangGerEng{Quelldatei (Basis)}{Sourcefile (parent)} & source/TUM/Modules/TUMbiblio.sty\\%
\hline%
\bfseries\RAIlangGerEng{Befehle}{Commands} & none\\\hline%
\end{tabular}\endgroup\par%
%
%
\subsection*{\RAIlangGerEng{Beschreibung}{Description}}%
% >>> CONTENTS OF FILE source/TUM/Modules/TUMbiblio_doc.tex: <<<<<<<<<<<<<<<<<<<<<<<<<<<<<<<<<<<<<<<
% Documentation of file TUMbiblio.sty%
The \RAIdocumentationModule{RAIbiblio} module generates the layout of the citations and bibliography. You can choose between the two backends \textit{biber} (\RAIdocumentationOption{optBiber}, modern, recommended) and \textit{bibtex} (\RAIdocumentationOption{optBibtex}, outdated, not recommended). Note, that the module is only loaded, if a backend has been specified.\par%
%
Moreover the style of bibliography entries can be specified. An example for the references%
%
\begin{itemize}\itemsep0pt%
    \item Albert Einstein, ``Zur Elektrodynamik bewegter Körper'', 1905%
    \item Michel Goossens, Frank Mittelbach, and Alexander Samarin ``The LaTeX Companion'', 1993%
\end{itemize}%
%
is given for every option:%
%
\begin{itemize}%
    \item \RAIdocumentationOption{optBibstyleAuthorYear:} Einstein 1916 -- Goossens, Mittelbach and Samarin 1993%
    \item \RAIdocumentationOption{optBibstyleAlphabetic:} {[Ein16]} -- {[GMS93]}%
    \item \RAIdocumentationOption{optBibstyleNumeric:} [1] -- [2]%
\end{itemize}%
%
%
%
% >>> CONTENTS OF FILE source/IN/RAI/Modules/RAIbiblio_doc.tex: <<<<<<<<<<<<<<<<<<<<<<<<<<<<<<<<<<<<
% Documentation of file RAIbiblio.sty
%
%
%
%
%
\section{RAInames}%
\label{sec:module_RAInames}%
\begingroup\sffamily\begin{tabular}{|p{4cm}|p{11cm}|}%
\hline\bfseries\RAIlangGerEng{Abhängigkeiten}{Dependencies} & \hyperref[sec:module_RAIcore]{RAIcore} \\\hline%
\bfseries\RAIlangGerEng{Optionen}{Options} & none\\\hline%
\bfseries\RAIlangGerEng{Geladene Pakete}{Loaded Packages} & none\\\hline%
\bfseries\RAIlangGerEng{Quelldatei}{Sourcefile} & source/IN/RAI/Modules/RAInames.sty\\\hline%
\bfseries\RAIlangGerEng{Quelldatei (Basis)}{Sourcefile (parent)} & source/TUM/Modules/TUMnames.sty\\%
\hline%
\bfseries\RAIlangGerEng{Befehle}{Commands} & \scriptsize \hyperref[sec:command_RAInamesProfHerrmann]{\textbackslash RAInamesProfHerrmann} \hyperref[sec:command_RAInamesProfHofmann]{\textbackslash RAInamesProfHofmann} \hyperref[sec:command_RAInamesPresident]{\textbackslash RAInamesPresident} \hyperref[sec:command_RAInamesProfAdams]{\textbackslash RAInamesProfAdams} \hyperref[sec:command_RAInamesProfBengler]{\textbackslash RAInamesProfBengler} \hyperref[sec:command_RAInamesProfBerensmeier]{\textbackslash RAInamesProfBerensmeier} \hyperref[sec:command_RAInamesProfBottasso]{\textbackslash RAInamesProfBottasso} \hyperref[sec:command_RAInamesProfCaccamo]{\textbackslash RAInamesProfCaccamo} \hyperref[sec:command_RAInamesProfDrechsler]{\textbackslash RAInamesProfDrechsler} \hyperref[sec:command_RAInamesProfFottner]{\textbackslash RAInamesProfFottner} \hyperref[sec:command_RAInamesProfGee]{\textbackslash RAInamesProfGee} \hyperref[sec:command_RAInamesProfGrosse]{\textbackslash RAInamesProfGrosse} \hyperref[sec:command_RAInamesProfGuemmer]{\textbackslash RAInamesProfGuemmer} \hyperref[sec:command_RAInamesProfGuenthner]{\textbackslash RAInamesProfGuenthner} \hyperref[sec:command_RAInamesProfHaidn]{\textbackslash RAInamesProfHaidn} \hyperref[sec:command_RAInamesProfHajek]{\textbackslash RAInamesProfHajek} \hyperref[sec:command_RAInamesProfHolzapfel]{\textbackslash RAInamesProfHolzapfel} \hyperref[sec:command_RAInamesProfHornung]{\textbackslash RAInamesProfHornung} \hyperref[sec:command_RAInamesProfKaltenbach]{\textbackslash RAInamesProfKaltenbach} \hyperref[sec:command_RAInamesProfKlein]{\textbackslash RAInamesProfKlein} \hyperref[sec:command_RAInamesProfKoutsourelakis]{\textbackslash RAInamesProfKoutsourelakis} \hyperref[sec:command_RAInamesProfKremling]{\textbackslash RAInamesProfKremling} \hyperref[sec:command_RAInamesProfLieleg]{\textbackslash RAInamesProfLieleg} \hyperref[sec:command_RAInamesProfLienkamp]{\textbackslash RAInamesProfLienkamp} \hyperref[sec:command_RAInamesProfLindemann]{\textbackslash RAInamesProfLindemann} \hyperref[sec:command_RAInamesProfLohmann]{\textbackslash RAInamesProfLohmann} \hyperref[sec:command_RAInamesProfLueth]{\textbackslash RAInamesProfLueth} \hyperref[sec:command_RAInamesProfMacianJuan]{\textbackslash RAInamesProfMacianJuan} \hyperref[sec:command_RAInamesProfMarburg]{\textbackslash RAInamesProfMarburg} \hyperref[sec:command_RAInamesProfNeu]{\textbackslash RAInamesProfNeu} \hyperref[sec:command_RAInamesProfPolifke]{\textbackslash RAInamesProfPolifke} \hyperref[sec:command_RAInamesProfProvost]{\textbackslash RAInamesProfProvost} \hyperref[sec:command_RAInamesProfReinhart]{\textbackslash RAInamesProfReinhart} \hyperref[sec:command_RAInamesProfRixen]{\textbackslash RAInamesProfRixen} \hyperref[sec:command_RAInamesProfSattelmayer]{\textbackslash RAInamesProfSattelmayer} \hyperref[sec:command_RAInamesProfSenner]{\textbackslash RAInamesProfSenner} \hyperref[sec:command_RAInamesProfSpannerUlmer]{\textbackslash RAInamesProfSpannerUlmer} \hyperref[sec:command_RAInamesProfSpliethoff]{\textbackslash RAInamesProfSpliethoff} \hyperref[sec:command_RAInamesProfStahl]{\textbackslash RAInamesProfStahl} \hyperref[sec:command_RAInamesProfVogelHeuser]{\textbackslash RAInamesProfVogelHeuser} \hyperref[sec:command_RAInamesProfVolk]{\textbackslash RAInamesProfVolk} \hyperref[sec:command_RAInamesProfWachtmeister]{\textbackslash RAInamesProfWachtmeister} \hyperref[sec:command_RAInamesProfWall]{\textbackslash RAInamesProfWall} \hyperref[sec:command_RAInamesProfWalter]{\textbackslash RAInamesProfWalter} \hyperref[sec:command_RAInamesProfWerner]{\textbackslash RAInamesProfWerner} \hyperref[sec:command_RAInamesProfWeusterBotz]{\textbackslash RAInamesProfWeusterBotz} \hyperref[sec:command_RAInamesProfZaeh]{\textbackslash RAInamesProfZaeh} \hyperref[sec:command_RAInamesProfZimmermann]{\textbackslash RAInamesProfZimmermann} \hyperref[sec:command_RAInamesProfAlthoff]{\textbackslash RAInamesProfAlthoff} \hyperref[sec:command_RAInamesProfBurschka]{\textbackslash RAInamesProfBurschka} \hyperref[sec:command_RAInamesProfKnoll]{\textbackslash RAInamesProfKnoll} \hyperref[sec:command_RAInamesProfPretschner]{\textbackslash RAInamesProfPretschner} \\\hline%
\end{tabular}\endgroup\par%
%
%
\subsection*{\RAIlangGerEng{Beschreibung}{Description}}%
% >>> CONTENTS OF FILE source/TUM/Modules/TUMnames_doc.tex: <<<<<<<<<<<<<<<<<<<<<<<<<<<<<<<<<<<<<<<<
% Documentation of file TUMnames.sty%
The \RAIdocumentationModule{RAInames} module implements the names of various active professors (see \cref{sec:commands}).\par%
%
%
%
% >>> CONTENTS OF FILE source/IN/RAI/Modules/RAInames_doc.tex: <<<<<<<<<<<<<<<<<<<<<<<<<<<<<<<<<<<<<
% Documentation of file RAInames.sty
%
%
%
%
%
\section{RAIutils}%
\label{sec:module_RAIutils}%
\begingroup\sffamily\begin{tabular}{|p{4cm}|p{11cm}|}%
\hline\bfseries\RAIlangGerEng{Abhängigkeiten}{Dependencies} & \hyperref[sec:module_RAIcore]{RAIcore} \hyperref[sec:module_RAIlang]{RAIlang} \hyperref[sec:module_RAIcolor]{RAIcolor} \\\hline%
\bfseries\RAIlangGerEng{Optionen}{Options} & optHideTodos optHideAnnotations \\\hline%
\bfseries\RAIlangGerEng{Geladene Pakete}{Loaded Packages} & excludeonly epstopdf xspace cancel datetime eurosym blindtext titlesec titletoc pdfcomment \\\hline%
\bfseries\RAIlangGerEng{Quelldatei}{Sourcefile} & source/IN/RAI/Modules/RAIutils.sty\\\hline%
\bfseries\RAIlangGerEng{Quelldatei (Basis)}{Sourcefile (parent)} & source/TUM/Modules/TUMutils.sty\\%
\hline%
\bfseries\RAIlangGerEng{Befehle}{Commands} & \scriptsize \hyperref[sec:command_RAIutilsTodo]{\textbackslash RAIutilsTodo} \hyperref[sec:command_RAIutilsTodoW]{\textbackslash RAIutilsTodoW} \hyperref[sec:command_RAIutilsTodoSpacer]{\textbackslash RAIutilsTodoSpacer} \hyperref[sec:command_RAIutilsPDFanno]{\textbackslash RAIutilsPDFanno} \hyperref[sec:command_RAIutilsPDFannoColor]{\textbackslash RAIutilsPDFannoColor} \hyperref[sec:command_RAIutilsPDFhigh]{\textbackslash RAIutilsPDFhigh} \hyperref[sec:command_RAIutilsDate]{\textbackslash RAIutilsDate} \hyperref[sec:command_RAIutilsToday]{\textbackslash RAIutilsToday} \\\hline%
\end{tabular}\endgroup\par%
%
%
\subsection*{\RAIlangGerEng{Beschreibung}{Description}}%
% >>> CONTENTS OF FILE source/TUM/Modules/TUMutils_doc.tex: <<<<<<<<<<<<<<<<<<<<<<<<<<<<<<<<<<<<<<<<
% Documentation of file TUMutils.sty%
The \RAIdocumentationModule{RAIutils} module provides several utility functions, which make it easier to integrate comments, annotations and other notes into the final PDF.\par%
%
%
\subsection*{Todos}%
There are several ways to typeset todos:
%
\begin{center}%
    \begin{tabular}{|lll|}%
        \hline%
        \textbf{Command} & \textbf{Output} & \textbf{Comment}\\\hline%
        \hyperref[sec:command_RAIutilsTodo]{\textbackslash RAIutilsTodo} & \RAIutilsTodo{default}, \RAIutilsTodo[TUMBlue]{custom} & simple todo note (color customizable)\\%
        \hyperref[sec:command_RAIutilsTodoW]{\textbackslash RAIutilsTodoW} & \RAIutilsTodo{default}, \RAIutilsTodo[TUMBlue]{custom} & todo note with LaTeX warning\\%
        \hline%
    \end{tabular}%
\end{center}%
%
There is also the possibility to insert a placeholder by \hyperref[sec:command_RAIutilsTodoSpacer]{\textbackslash RAIutilsTodoSpacer}:%
\RAIutilsTodoSpacer[custom spacer header]{15}%
%
\textbf{Note:} Todos can be hidden globally by passing the option \RAIdocumentationOption{optHideTodos}.%
%
\subsection*{Comments and annotations}%
If the document length should not be influenced by a comment, one may use the following commands:%
\begin{center}%
    \begin{tabular}{|lll|}%
        \hline%
        \textbf{Command} & \textbf{Output} & \textbf{Comment}\\\hline%
        \hyperref[sec:command_RAIutilsPDFanno]{\textbackslash RAIutilsPDFanno} & annotated text\RAIutilsPDFanno[author name]{annotation text} & simple pdf popup annotation\\%
        \hyperref[sec:command_RAIutilsPDFannoColor]{\textbackslash RAIutilsPDFannoColor} & annotated text\RAIutilsPDFannoColor[author name]{annotation text}{TUMBlue} & pdf popup annotation (custom color)\\%
        \hyperref[sec:command_RAIutilsPDFhigh]{\textbackslash RAIutilsPDFhigh} & \RAIutilsPDFhigh[yellow]{highlighted text passage} & pdf highlighted text (customizable color)\\%
        \hline%
    \end{tabular}%
\end{center}%
%
\textbf{Note:} Comments and annotations can be hidden globally by passing the option \RAIdocumentationOption{optHideAnnotations}.%
%
\subsection*{Dates}%
In order to typeset dates, the two commands \hyperref[sec:command_RAIutilsDate]{\textbackslash RAIutilsDate} and \hyperref[sec:command_RAIutilsToday]{\textbackslash RAIutilsToday} exist.%
\begin{center}%
    \begin{tabular}{|lll|}%
        \hline%
        \textbf{Command} & \textbf{Output} & \textbf{Comment}\\\hline%
        \hyperref[sec:command_RAIutilsDate]{\textbackslash RAIutilsDate} & \RAIutilsDate{1}{1}{1970} & displays a custom date\\%
        \hyperref[sec:command_RAIutilsToday]{\textbackslash RAIutilsToday} & \RAIutilsToday & displays the current date (compile time)\\%
        \hline%
    \end{tabular}%
\end{center}%
%
%
%
% >>> CONTENTS OF FILE source/IN/RAI/Modules/RAIutils_doc.tex: <<<<<<<<<<<<<<<<<<<<<<<<<<<<<<<<<<<<<
% Documentation of file RAIutils.sty
%
%
%
%
%
\section{RAIref}%
\label{sec:module_RAIref}%
\begingroup\sffamily\begin{tabular}{|p{4cm}|p{11cm}|}%
\hline\bfseries\RAIlangGerEng{Abhängigkeiten}{Dependencies} & \hyperref[sec:module_RAIcore]{RAIcore} \hyperref[sec:module_RAIlang]{RAIlang} \hyperref[sec:module_RAIcolor]{RAIcolor} \\\hline%
\bfseries\RAIlangGerEng{Optionen}{Options} & optBlackRefs \\\hline%
\bfseries\RAIlangGerEng{Geladene Pakete}{Loaded Packages} & hyperref cleveref \\\hline%
\bfseries\RAIlangGerEng{Quelldatei}{Sourcefile} & source/IN/RAI/Modules/RAIref.sty\\\hline%
\bfseries\RAIlangGerEng{Quelldatei (Basis)}{Sourcefile (parent)} & source/TUM/Modules/TUMref.sty\\%
\hline%
\bfseries\RAIlangGerEng{Befehle}{Commands} & \scriptsize \hyperref[sec:command_RAIrefSetPDFMetadata]{\textbackslash RAIrefSetPDFMetadata} \\\hline%
\end{tabular}\endgroup\par%
%
%
\subsection*{\RAIlangGerEng{Beschreibung}{Description}}%
% >>> CONTENTS OF FILE source/TUM/Modules/TUMref_doc.tex: <<<<<<<<<<<<<<<<<<<<<<<<<<<<<<<<<<<<<<<<<<
% Documentation of file TUMref.sty%
The \RAIdocumentationModule{RAIref} module takes care of in-document referencing. Basically it loads and initializes the \RAIdocumentationCode{hyperref} and \RAIdocumentationCode{cleveref} packages. Furthermore the module provides a command to set the PDF meta-data (properties of the pdf document).\par%
You can use the \RAIdocumentationOption{optBlackRefs} to disable colored references.
%
\textbf{Hint:} This module loads the \RAIdocumentationCode{hyperref} packages which often has to be loaded \textbf{after} loading certain other packages. In order to load packages \textbf{before} \RAIdocumentationCode{hyperref} and \textbf{without} modifying the TUMlatex class/package you can add a file \RAIdocumentationCode{pre\_hyperref\_packages.tex} right beside your \RAIdocumentationCode{main.tex} file. The TUMlatex package will automatically execute the content of this file (i.\,e. your additional package loading macros) directly before loading the \RAIdocumentationCode{hyperref} package.\par%
%
%
%
% >>> CONTENTS OF FILE source/IN/RAI/Modules/RAIref_doc.tex: <<<<<<<<<<<<<<<<<<<<<<<<<<<<<<<<<<<<<<<
% Documentation of file RAIref.sty
%
%
%
%
%

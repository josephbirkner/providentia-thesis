% Auto-generated style documentation
\section{RAIphdprepublication}%
\label{sec:style_RAIphdprepublication}%
\begingroup\sffamily\begin{tabular}{|p{4cm}|p{11cm}|}%
\hline\bfseries\RAIlangGerEng{Basisklasse}{Baseclass} & [11pt,a4paper,fleqn]{article}\\\hline%
\bfseries\RAIlangGerEng{Optionen}{Options} & none\\\hline%
\bfseries\RAIlangGerEng{Geladene Pakete}{Loaded Packages} & caption fancyhdr \\\hline%
\bfseries\RAIlangGerEng{Quelldatei}{Sourcefile} & source/IN/RAI/Styles/RAIphdprepublication.sty\\\hline%
\bfseries\RAIlangGerEng{Quelldatei (Basis)}{Sourcefile (parent)} & source/TUM/Styles/TUMphdprepublication.sty\\%
\hline%
\bfseries\RAIlangGerEng{Befehle}{Commands} & \scriptsize \hyperref[sec:command_RAIphdprepublicationTitlePage]{\textbackslash RAIphdprepublicationTitlePage} \\\hline%
\end{tabular}\endgroup\par%
%
%
\subsection*{\RAIlangGerEng{Beschreibung}{Description}}%
% >>> CONTENTS OF FILE source/TUM/Styles/TUMphdprepublication_doc.tex: <<<<<<<<<<<<<<<<<<<<<<<<<<<<<
% Documentation of file TUMphdprepublication.sty%
The style \RAIdocumentationStyle{RAIphdprepublication} helps PhD students to set up their pre-publication list for the TUM Graduate School (at the end of the doctoral project). The corresponding template \RAIdocumentationTemplate{RAIphdprepublication} is approved by the Graduate School.\par%
%
%
%
%
%
\section{RAIempty}%
\label{sec:style_RAIempty}%
\begingroup\sffamily\begin{tabular}{|p{4cm}|p{11cm}|}%
\hline\bfseries\RAIlangGerEng{Basisklasse}{Baseclass} & [11pt,a4paper,fleqn]{article}\\\hline%
\bfseries\RAIlangGerEng{Optionen}{Options} & none\\\hline%
\bfseries\RAIlangGerEng{Geladene Pakete}{Loaded Packages} & none\\\hline%
\bfseries\RAIlangGerEng{Quelldatei}{Sourcefile} & source/IN/RAI/Styles/RAIempty.sty\\\hline%
\bfseries\RAIlangGerEng{Quelldatei (Basis)}{Sourcefile (parent)} & source/TUM/Styles/TUMempty.sty\\%
\hline%
\bfseries\RAIlangGerEng{Befehle}{Commands} & none\\\hline%
\end{tabular}\endgroup\par%
%
%
\subsection*{\RAIlangGerEng{Beschreibung}{Description}}%
% >>> CONTENTS OF FILE source/TUM/Styles/TUMempty_doc.tex: <<<<<<<<<<<<<<<<<<<<<<<<<<<<<<<<<<<<<<<<<
% Documentation of file TUMempty.sty%
The \RAIdocumentationStyle{RAIempty} style is probably the most difficult one to use, since it is a completely empty style. That means the complete formatting process is up to the user. An unexperienced user may therefore choose another style to start with.\par%
%
%
%
% >>> CONTENTS OF FILE source/IN/RAI/Styles/RAIempty_doc.tex: <<<<<<<<<<<<<<<<<<<<<<<<<<<<<<<<<<<<<<
% Documentation of file RAIempty.sty
%
%
%
%
%
\section{RAIphdexpose}%
\label{sec:style_RAIphdexpose}%
\begingroup\sffamily\begin{tabular}{|p{4cm}|p{11cm}|}%
\hline\bfseries\RAIlangGerEng{Basisklasse}{Baseclass} & [11pt,a4paper,fleqn]{article}\\\hline%
\bfseries\RAIlangGerEng{Optionen}{Options} & none\\\hline%
\bfseries\RAIlangGerEng{Geladene Pakete}{Loaded Packages} & caption fancyhdr \\\hline%
\bfseries\RAIlangGerEng{Quelldatei}{Sourcefile} & source/IN/RAI/Styles/RAIphdexpose.sty\\\hline%
\bfseries\RAIlangGerEng{Quelldatei (Basis)}{Sourcefile (parent)} & source/TUM/Styles/TUMphdexpose.sty\\%
\hline%
\bfseries\RAIlangGerEng{Befehle}{Commands} & \scriptsize \hyperref[sec:command_RAIphdexposeTitlePage]{\textbackslash RAIphdexposeTitlePage} \\\hline%
\end{tabular}\endgroup\par%
%
%
\subsection*{\RAIlangGerEng{Beschreibung}{Description}}%
% >>> CONTENTS OF FILE source/TUM/Styles/TUMphdexpose_doc.tex: <<<<<<<<<<<<<<<<<<<<<<<<<<<<<<<<<<<<<
% Documentation of file TUMphdexpose.sty%
The style \RAIdocumentationStyle{RAIphdexpose} helps PhD students to set up their expose for the TUM Graduate School. The corresponding template \RAIdocumentationTemplate{RAIphdexpose} is approved by the Graduate School.\par%
%
%
%
% >>> CONTENTS OF FILE source/IN/RAI/Styles/RAIphdexpose_doc.tex: <<<<<<<<<<<<<<<<<<<<<<<<<<<<<<<<<<
% Documentation of file RAIphdexpose.sty
%
%
%
%
%
\section{RAIphdthesis}%
\label{sec:style_RAIphdthesis}%
\begingroup\sffamily\begin{tabular}{|p{4cm}|p{11cm}|}%
\hline\bfseries\RAIlangGerEng{Basisklasse}{Baseclass} & [11pt,a4paper,twoside]{book}\\\hline%
\bfseries\RAIlangGerEng{Optionen}{Options} & none\\\hline%
\bfseries\RAIlangGerEng{Geladene Pakete}{Loaded Packages} & caption fancyhdr \\\hline%
\bfseries\RAIlangGerEng{Quelldatei}{Sourcefile} & source/IN/RAI/Styles/RAIphdthesis.sty\\\hline%
\bfseries\RAIlangGerEng{Quelldatei (Basis)}{Sourcefile (parent)} & source/TUM/Styles/TUMphdthesis.sty\\%
\hline%
\bfseries\RAIlangGerEng{Befehle}{Commands} & \scriptsize \hyperref[sec:command_RAIphdthesisPrintTableOfContents]{\textbackslash RAIphdthesisPrintTableOfContents} \hyperref[sec:command_RAIphdthesisTitlePage]{\textbackslash RAIphdthesisTitlePage} \\\hline%
\end{tabular}\endgroup\par%
%
%
\subsection*{\RAIlangGerEng{Beschreibung}{Description}}%
% >>> CONTENTS OF FILE source/TUM/Styles/TUMphdthesis_doc.tex: <<<<<<<<<<<<<<<<<<<<<<<<<<<<<<<<<<<<<
% Documentation of file TUMphdthesis.sty%
The \RAIdocumentationStyle{RAIphdthesis} style helps research assistants to write their PhD thesis document.\par%
%
%
%
% >>> CONTENTS OF FILE source/IN/RAI/Styles/RAIphdthesis_doc.tex: <<<<<<<<<<<<<<<<<<<<<<<<<<<<<<<<<<
% Documentation of file RAIphdthesis.sty
%
%
%
%
%
\section{RAIdocumentation}%
\label{sec:style_RAIdocumentation}%
\begingroup\sffamily\begin{tabular}{|p{4cm}|p{11cm}|}%
\hline\bfseries\RAIlangGerEng{Basisklasse}{Baseclass} & [11pt,a4paper,oneside,fleqn]{book}\\\hline%
\bfseries\RAIlangGerEng{Optionen}{Options} & none\\\hline%
\bfseries\RAIlangGerEng{Geladene Pakete}{Loaded Packages} & caption longtable fancyhdr \\\hline%
\bfseries\RAIlangGerEng{Quelldatei}{Sourcefile} & source/IN/RAI/Styles/RAIdocumentation.sty\\\hline%
\bfseries\RAIlangGerEng{Quelldatei (Basis)}{Sourcefile (parent)} & source/TUM/Styles/TUMdocumentation.sty\\%
\hline%
\bfseries\RAIlangGerEng{Befehle}{Commands} & \scriptsize \hyperref[sec:command_RAIdocumentationModule]{\textbackslash RAIdocumentationModule} \hyperref[sec:command_RAIdocumentationStyle]{\textbackslash RAIdocumentationStyle} \hyperref[sec:command_RAIdocumentationTemplate]{\textbackslash RAIdocumentationTemplate} \hyperref[sec:command_RAIdocumentationOption]{\textbackslash RAIdocumentationOption} \hyperref[sec:command_RAIdocumentationCode]{\textbackslash RAIdocumentationCode} \\\hline%
\end{tabular}\endgroup\par%
%
%
\subsection*{\RAIlangGerEng{Beschreibung}{Description}}%
% >>> CONTENTS OF FILE source/TUM/Styles/TUMdocumentation_doc.tex: <<<<<<<<<<<<<<<<<<<<<<<<<<<<<<<<<
% Documentation of file TUMdocumentation.sty%
The \RAIdocumentationStyle{RAIdocumentation} style is used to specify the layout of \textbf{this} document. There are several commands to typeset special names:%
%
\begin{center}%
    \begin{tabular}{|lll|}%
        \hline%
        \textbf{Command} & \textbf{Output} & \textbf{Description}\\\hline%
        \hyperref[sec:command_RAIdocumentationModule]{\textbackslash RAIdocumentationModule} & \RAIdocumentationModule{module} & typesets module names\\%
        \hyperref[sec:command_RAIdocumentationStyle]{\textbackslash RAIdocumentationStyle} & \RAIdocumentationStyle{style} & typesets style names\\%
        \hyperref[sec:command_RAIdocumentationTemplate]{\textbackslash RAIdocumentationTemplate} & \RAIdocumentationTemplate{template} & typesets template names\\%
        \hyperref[sec:command_RAIdocumentationOption]{\textbackslash RAIdocumentationOption} & \RAIdocumentationOption{option} & typesets options\\%
        \hyperref[sec:command_RAIdocumentationCode]{\textbackslash RAIdocumentationCode} & \RAIdocumentationCode{code} & typesets LaTeX code\\%
        \hline%
    \end{tabular}%
\end{center}%
%
%
%
% >>> CONTENTS OF FILE source/IN/RAI/Styles/RAIdocumentation_doc.tex: <<<<<<<<<<<<<<<<<<<<<<<<<<<<<<
% Documentation of file RAIdocumentation.sty
%
%
%
%
%
\section{RAIbeamer}%
\label{sec:style_RAIbeamer}%
\begingroup\sffamily\begin{tabular}{|p{4cm}|p{11cm}|}%
\hline\bfseries\RAIlangGerEng{Basisklasse}{Baseclass} & [noamssymb,t,11pt]{beamer}\\\hline%
\bfseries\RAIlangGerEng{Optionen}{Options} & none\\\hline%
\bfseries\RAIlangGerEng{Geladene Pakete}{Loaded Packages} & caption transparent \\\hline%
\bfseries\RAIlangGerEng{Quelldatei}{Sourcefile} & source/IN/RAI/Styles/RAIbeamer.sty\\\hline%
\bfseries\RAIlangGerEng{Quelldatei (Basis)}{Sourcefile (parent)} & source/TUM/Styles/TUMbeamer.sty\\%
\hline%
\bfseries\RAIlangGerEng{Befehle}{Commands} & \scriptsize \hyperref[sec:command_RAIbeamerTitlePageStudentThesis]{\textbackslash RAIbeamerTitlePageStudentThesis} \hyperref[sec:command_RAIbeamerSetupHeader]{\textbackslash RAIbeamerSetupHeader} \hyperref[sec:command_RAIbeamerSetFooterText]{\textbackslash RAIbeamerSetFooterText} \hyperref[sec:command_RAIbeamerSetupFooterCD]{\textbackslash RAIbeamerSetupFooterCD} \hyperref[sec:command_RAIbeamerSetupFooterSlideNumberOnly]{\textbackslash RAIbeamerSetupFooterSlideNumberOnly} \hyperref[sec:command_RAIbeamerTitlePageDefault]{\textbackslash RAIbeamerTitlePageDefault} \\\hline%
\end{tabular}\endgroup\par%
%
%
\subsection*{\RAIlangGerEng{Beschreibung}{Description}}%
% >>> CONTENTS OF FILE source/TUM/Styles/TUMbeamer_doc.tex: <<<<<<<<<<<<<<<<<<<<<<<<<<<<<<<<<<<<<<<<
% Documentation of file TUMbeamer.sty%
The \RAIdocumentationStyle{RAIbeamer} style creates a default style for presentations. It can be used for lectures, conference presentations or student thesis presentations. The style complies in general with the TUM corporate design, such that it can also be used for external presentations (e.\,g. conferences).\par%
%
To customize the colors, fonts, font sizes or even complete templates one can use the standard beamer macros%
\begin{itemize}%
  \item \RAIdocumentationCode{\textbackslash setbeamercolor\{template name\}\{<key=value> list\}},%
  \item \RAIdocumentationCode{\textbackslash setbeamerfont\{template name\}\{<key=value> list\}},%
  \item \RAIdocumentationCode{\textbackslash setbeamersize\{size name=<dim>\}},%
  \item \RAIdocumentationCode{\textbackslash setbeamertemplate\{template name\}\{your definition\}}.%
\end{itemize}%
%
There is a useful cheat-sheet on how to modify the style of your beamer presentation available online:\par%
\url{http://www.cpt.univ-mrs.fr/~masson/latex/Beamer-appearance-cheat-sheet.pdf}
%
%
%
% >>> CONTENTS OF FILE source/IN/RAI/Styles/RAIbeamer_doc.tex: <<<<<<<<<<<<<<<<<<<<<<<<<<<<<<<<<<<<<
% Documentation of file RAIbeamer.sty
%
%
%
%
%
\section{RAIstudentexpose}%
\label{sec:style_RAIstudentexpose}%
\begingroup\sffamily\begin{tabular}{|p{4cm}|p{11cm}|}%
\hline\bfseries\RAIlangGerEng{Basisklasse}{Baseclass} & [11pt,a4paper,fleqn]{article}\\\hline%
\bfseries\RAIlangGerEng{Optionen}{Options} & none\\\hline%
\bfseries\RAIlangGerEng{Geladene Pakete}{Loaded Packages} & caption fancyhdr \\\hline%
\bfseries\RAIlangGerEng{Quelldatei}{Sourcefile} & source/IN/RAI/Styles/RAIstudentexpose.sty\\\hline%
\bfseries\RAIlangGerEng{Quelldatei (Basis)}{Sourcefile (parent)} & source/TUM/Styles/TUMstudentexpose.sty\\%
\hline%
\bfseries\RAIlangGerEng{Befehle}{Commands} & \scriptsize \hyperref[sec:command_RAIstudentexposeTitlePage]{\textbackslash RAIstudentexposeTitlePage} \hyperref[sec:command_RAIstudentexposeHeader]{\textbackslash RAIstudentexposeHeader} \hyperref[sec:command_RAIstudentexposeTitlePageBachelorsThesis]{\textbackslash RAIstudentexposeTitlePageBachelorsThesis} \hyperref[sec:command_RAIstudentexposeTitlePageSemesterThesis]{\textbackslash RAIstudentexposeTitlePageSemesterThesis} \hyperref[sec:command_RAIstudentexposeTitlePageMastersThesis]{\textbackslash RAIstudentexposeTitlePageMastersThesis} \hyperref[sec:command_RAIstudentexposeTitlePageIDP]{\textbackslash RAIstudentexposeTitlePageIDP} \hyperref[sec:command_RAIstudentexposeReview]{\textbackslash RAIstudentexposeReview} \\\hline%
\end{tabular}\endgroup\par%
%
%
\subsection*{\RAIlangGerEng{Beschreibung}{Description}}%
% >>> CONTENTS OF FILE source/TUM/Styles/TUMstudentexpose_doc.tex: <<<<<<<<<<<<<<<<<<<<<<<<<<<<<<<<<
% Documentation of file TUMstudentexpose.sty%
The style \RAIdocumentationStyle{RAIstudentexpose} helps students to set up their expose for their student thesis. The style is approved by the Center of Key Competencies of the TUM Department of Mechanical Engineering.\par%
%
%
%
% >>> CONTENTS OF FILE source/IN/RAI/Styles/RAIstudentexpose_doc.tex: <<<<<<<<<<<<<<<<<<<<<<<<<<<<<<
% Documentation of file RAIstudentexpose.sty%
%
%
%
%
%
\section{RAIletter}%
\label{sec:style_RAIletter}%
\begingroup\sffamily\begin{tabular}{|p{4cm}|p{11cm}|}%
\hline\bfseries\RAIlangGerEng{Basisklasse}{Baseclass} & [11pt,a4paper,fleqn]{article}\\\hline%
\bfseries\RAIlangGerEng{Optionen}{Options} & none\\\hline%
\bfseries\RAIlangGerEng{Geladene Pakete}{Loaded Packages} & lastpage \\\hline%
\bfseries\RAIlangGerEng{Quelldatei}{Sourcefile} & source/IN/RAI/Styles/RAIletter.sty\\\hline%
\bfseries\RAIlangGerEng{Quelldatei (Basis)}{Sourcefile (parent)} & source/TUM/Styles/TUMletter.sty\\%
\hline%
\bfseries\RAIlangGerEng{Befehle}{Commands} & \scriptsize \hyperref[sec:command_RAIletterSetReceiver]{\textbackslash RAIletterSetReceiver} \hyperref[sec:command_RAIletterSetPlace]{\textbackslash RAIletterSetPlace} \hyperref[sec:command_RAIletterSetDate]{\textbackslash RAIletterSetDate} \hyperref[sec:command_RAIletterSetSubject]{\textbackslash RAIletterSetSubject} \hyperref[sec:command_RAIletterSetTitle]{\textbackslash RAIletterSetTitle} \hyperref[sec:command_RAIletterSetName]{\textbackslash RAIletterSetName} \hyperref[sec:command_RAIletterSetPhone]{\textbackslash RAIletterSetPhone} \hyperref[sec:command_RAIletterSetFax]{\textbackslash RAIletterSetFax} \hyperref[sec:command_RAIletterSetEMail]{\textbackslash RAIletterSetEMail} \hyperref[sec:command_RAIletterSetHomepage]{\textbackslash RAIletterSetHomepage} \hyperref[sec:command_RAIletterHeader]{\textbackslash RAIletterHeader} \\\hline%
\end{tabular}\endgroup\par%
%
%
\subsection*{\RAIlangGerEng{Beschreibung}{Description}}%
% >>> CONTENTS OF FILE source/TUM/Styles/TUMletter_doc.tex: <<<<<<<<<<<<<<<<<<<<<<<<<<<<<<<<<<<<<<<<
% Documentation of file TUMletter.sty%
The \RAIdocumentationStyle{RAIletter} style typesets a default letter according to the TUM Corporate Design.\par%
%
%
%
% >>> CONTENTS OF FILE source/IN/RAI/Styles/RAIletter_doc.tex: <<<<<<<<<<<<<<<<<<<<<<<<<<<<<<<<<<<<<
% Documentation of file RAIletter.sty
%
%
%
%
%
\section{RAIposter}%
\label{sec:style_RAIposter}%
\begingroup\sffamily\begin{tabular}{|p{4cm}|p{11cm}|}%
\hline\bfseries\RAIlangGerEng{Basisklasse}{Baseclass} & [11pt,a4paper,fleqn]{article}\\\hline%
\bfseries\RAIlangGerEng{Optionen}{Options} & none\\\hline%
\bfseries\RAIlangGerEng{Geladene Pakete}{Loaded Packages} & none\\\hline%
\bfseries\RAIlangGerEng{Quelldatei}{Sourcefile} & source/IN/RAI/Styles/RAIposter.sty\\\hline%
\bfseries\RAIlangGerEng{Quelldatei (Basis)}{Sourcefile (parent)} & source/TUM/Styles/TUMposter.sty\\%
\hline%
\bfseries\RAIlangGerEng{Befehle}{Commands} & \scriptsize \hyperref[sec:command_RAIposterDrawGrid]{\textbackslash RAIposterDrawGrid} \\\hline%
\end{tabular}\endgroup\par%
%
%
\subsection*{\RAIlangGerEng{Beschreibung}{Description}}%
% >>> CONTENTS OF FILE source/TUM/Styles/TUMposter_doc.tex: <<<<<<<<<<<<<<<<<<<<<<<<<<<<<<<<<<<<<<<<
% Documentation of file TUMposter.sty%
The \RAIdocumentationStyle{RAIposter} style is used to create individual posters of size DIN A0. In order to comply with the poster size, the font size, logos etc. are scaled.\par%
%
%
%
% >>> CONTENTS OF FILE source/IN/RAI/Styles/RAIposter_doc.tex: <<<<<<<<<<<<<<<<<<<<<<<<<<<<<<<<<<<<<
% Documentation of file RAIposter.sty
%
%
%
%
%
\section{RAIstudentthesis}%
\label{sec:style_RAIstudentthesis}%
\begingroup\sffamily\begin{tabular}{|p{4cm}|p{11cm}|}%
\hline\bfseries\RAIlangGerEng{Basisklasse}{Baseclass} & [11pt,a4paper,twoside,fleqn]{book}\\\hline%
\bfseries\RAIlangGerEng{Optionen}{Options} & optBlackHeadings \\\hline%
\bfseries\RAIlangGerEng{Geladene Pakete}{Loaded Packages} & caption fancyhdr ifthen \\\hline%
\bfseries\RAIlangGerEng{Quelldatei}{Sourcefile} & source/IN/RAI/Styles/RAIstudentthesis.sty\\\hline%
\bfseries\RAIlangGerEng{Quelldatei (Basis)}{Sourcefile (parent)} & source/TUM/Styles/TUMstudentthesis.sty\\%
\hline%
\bfseries\RAIlangGerEng{Befehle}{Commands} & \scriptsize \hyperref[sec:command_RAITitlePageStudentThesisCustomDiplomarbeit]{\textbackslash RAITitlePageStudentThesisCustomDiplomarbeit} \hyperref[sec:command_RAIstudentthesisTitlePageCustomBachelorsThesis]{\textbackslash RAIstudentthesisTitlePageCustomBachelorsThesis} \hyperref[sec:command_RAIstudentthesisTitlePageCustomSemesterThesis]{\textbackslash RAIstudentthesisTitlePageCustomSemesterThesis} \hyperref[sec:command_RAIstudentthesisTitlePageCustomMastersThesis]{\textbackslash RAIstudentthesisTitlePageCustomMastersThesis} \hyperref[sec:command_RAIstudentthesisTitlePageCustomIDP]{\textbackslash RAIstudentthesisTitlePageCustomIDP} \hyperref[sec:command_RAIstudentthesisAbstract]{\textbackslash RAIstudentthesisAbstract} \hyperref[sec:command_RAIstudentthesisPrintTableOfContents]{\textbackslash RAIstudentthesisPrintTableOfContents} \hyperref[sec:command_RAIstudentthesisTitlePageCD]{\textbackslash RAIstudentthesisTitlePageCD} \hyperref[sec:command_RAIstudentthesisTitlePageFrontPage]{\textbackslash RAIstudentthesisTitlePageFrontPage} \hyperref[sec:command_RAIstudentthesisTitlePageFrontPageInfoTable]{\textbackslash RAIstudentthesisTitlePageFrontPageInfoTable} \hyperref[sec:command_RAIstudentthesisDeclaration]{\textbackslash RAIstudentthesisDeclaration} \hyperref[sec:command_RAIstudentthesisTitlePageCDDiplomarbeit]{\textbackslash RAIstudentthesisTitlePageCDDiplomarbeit} \hyperref[sec:command_RAIstudentthesisTitlePageCDBachelorsThesis]{\textbackslash RAIstudentthesisTitlePageCDBachelorsThesis} \hyperref[sec:command_RAIstudentthesisTitlePageCDSemesterThesis]{\textbackslash RAIstudentthesisTitlePageCDSemesterThesis} \hyperref[sec:command_RAIstudentthesisTitlePageCDMasterThesis]{\textbackslash RAIstudentthesisTitlePageCDMasterThesis} \hyperref[sec:command_RAIstudentthesisTitlePageCDIDP]{\textbackslash RAIstudentthesisTitlePageCDIDP} \\\hline%
\end{tabular}\endgroup\par%
%
%
\subsection*{\RAIlangGerEng{Beschreibung}{Description}}%
% >>> CONTENTS OF FILE source/TUM/Styles/TUMstudentthesis_doc.tex: <<<<<<<<<<<<<<<<<<<<<<<<<<<<<<<<<
% Documentation of file TUMstudentthesis.sty%
The \RAIdocumentationStyle{RAIstudentthesis} style was created to support students by providing unified thesis templates.\par%
%
%
%
% >>> CONTENTS OF FILE source/IN/RAI/Styles/RAIstudentthesis_doc.tex: <<<<<<<<<<<<<<<<<<<<<<<<<<<<<<
% Documentation of file RAIstudentthesis.sty
%
%
The \RAIdocumentationOption{optBlackHeadings} option disables colored section styles to reduce the number of colored pages.%
%
%
\section{RAIstudentthesiscover}%
\label{sec:style_RAIstudentthesiscover}%
\begingroup\sffamily\begin{tabular}{|p{4cm}|p{11cm}|}%
\hline\bfseries\RAIlangGerEng{Basisklasse}{Baseclass} & [11pt,a4paper]{article}\\\hline%
\bfseries\RAIlangGerEng{Optionen}{Options} & optCoverWithMarks optCoverWithBorders \\\hline%
\bfseries\RAIlangGerEng{Geladene Pakete}{Loaded Packages} & calc \\\hline%
\bfseries\RAIlangGerEng{Quelldatei}{Sourcefile} & source/IN/RAI/Styles/RAIstudentthesiscover.sty\\\hline%
\bfseries\RAIlangGerEng{Quelldatei (Basis)}{Sourcefile (parent)} & source/TUM/Styles/TUMstudentthesiscover.sty\\%
\hline%
\bfseries\RAIlangGerEng{Befehle}{Commands} & \scriptsize \hyperref[sec:command_RAIstudentthesisCover]{\textbackslash RAIstudentthesisCover} \\\hline%
\end{tabular}\endgroup\par%
%
%
\subsection*{\RAIlangGerEng{Beschreibung}{Description}}%
% >>> CONTENTS OF FILE source/TUM/Styles/TUMstudentthesiscover_doc.tex: <<<<<<<<<<<<<<<<<<<<<<<<<<<<
% Documentation of file TUMstudentthesiscover.sty%
The \RAIdocumentationStyle{RAIstudentthesiscover} style defines a layout for covers of student theses (for printing). The count of PDF pages in the thesis is automatically determined to define the height of the spine.\par%
%
%
%
% >>> CONTENTS OF FILE source/IN/RAI/Styles/RAIstudentthesiscover_doc.tex: <<<<<<<<<<<<<<<<<<<<<<<<<
% Documentation of file RAIstudentthesiscover.sty%
%
%
%
%
%
\section{RAIdocument}%
\label{sec:style_RAIdocument}%
\begingroup\sffamily\begin{tabular}{|p{4cm}|p{11cm}|}%
\hline\bfseries\RAIlangGerEng{Basisklasse}{Baseclass} & [11pt,a4paper,fleqn]{article}\\\hline%
\bfseries\RAIlangGerEng{Optionen}{Options} & none\\\hline%
\bfseries\RAIlangGerEng{Geladene Pakete}{Loaded Packages} & caption \\\hline%
\bfseries\RAIlangGerEng{Quelldatei}{Sourcefile} & source/IN/RAI/Styles/RAIdocument.sty\\\hline%
\bfseries\RAIlangGerEng{Quelldatei (Basis)}{Sourcefile (parent)} & source/TUM/Styles/TUMdocument.sty\\%
\hline%
\bfseries\RAIlangGerEng{Befehle}{Commands} & none\\\hline%
\end{tabular}\endgroup\par%
%
%
\subsection*{\RAIlangGerEng{Beschreibung}{Description}}%
% >>> CONTENTS OF FILE source/TUM/Styles/TUMdocument_doc.tex: <<<<<<<<<<<<<<<<<<<<<<<<<<<<<<<<<<<<<<
% Documentation of file TUMdocument.sty%
The \RAIdocumentationStyle{RAIdocument} style defines a very basic layout. It can be used for most documents which do not fit other styles.\par%
%
%
%
% >>> CONTENTS OF FILE source/IN/RAI/Styles/RAIdocument_doc.tex: <<<<<<<<<<<<<<<<<<<<<<<<<<<<<<<<<<<
% Documentation of file RAIdocument.sty
%
%
%
%
%
